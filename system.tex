% todo - ctrl-f for ***

\documentclass[12pt]{report}
\large
\hbadness=99999

\usepackage{extsizes}
\usepackage{multido}
\usepackage{blindtext}
\usepackage{titlesec}
\usepackage{amssymb,amsfonts,amsmath}
\usepackage{mathrsfs}
\usepackage[unicode=true]{hyperref}
\usepackage[capitalise]{cleveref}
\usepackage{grbbridge}
\usepackage[utf8]{inputenc}
\usepackage{geometry}
\geometry{a4paper, total={170mm,257mm}, left=20mm, top=20mm,}

\newcommand{\n}{\\}
\newcommand{\q}[1]{\multido{}{#1}{\qquad}}
\newcommand{\ol}[1]{\begin{enumerate}#1\end{enumerate}}
\newcommand{\ul}[1]{\begin{itemize}#1\end{itemize}}
\newcommand{\li}[1]{\item[~] \q{#1}}
\newcommand{\bidsection}[2]{\section{\texorpdfstring{#1}{#2}}}


\title{\bf{2/1 Game Forcing}}
\author{Daniel Schuck, Han-Mo Ou}

%%%%%%%%%%%%%%%%%%%%%%%%%%%%%%%%%%%%%%%%%%%%%%%%%%%%%%%%%%%%%%%%%%%%%
%%%%%%%%%%%%%%%%%%%%%%%%%%%%%%%%%%%%%%%%%%%%%%%%%%%%%%%%%%%%%%%%%%%%%
%%%%%%%%%%%%%%%%%%%%%%%%%%%%%%%%%%%%%%%%%%%%%%%%%%%%%%%%%%%%%%%%%%%%%

\begin{document}

\maketitle

\tableofcontents

\chapter{Introduction} \label{1}
\section{Biography}

    Daniel Schuck, a current junior at University of Illinois Urbana-Champaign.  An avid overbidder and the primary author of this book.  System and convention design is fascinating to me, despite cardplay being far more important.
\n

\section{Structure}

    This system is built off a fairly standard 2/1 game-forcing system.  Major modifications include Gazzilli over 1M, shape-showing responses when seeking 6m after 1N opening. \n
    
    Two common motifs are used for artificial raises: \n
    
    The first motif: if one must decide between showing a major fit and a minor fit, show the major first. For instance, after 1NT-\di2-\he2-\cl3, a club fit is confirmed via \di3, which necessarily \textit{denies} a heart fit, as opener must show the major fit first.  By confirming a major suit, this suit is \textit{flagged immediately} as the trump suit, and the partnership is committed to playing in either that major or notrumps (an auction such as \sp1-\he2-\he3-\sp3 is a cuebid since \he3 flags hearts, not support-showing).  An example of playing in NT after a major fit is when a choice of games is offered: \cl1-\he1-\he2-3NT.  \n

    The second motif is surrogacy.  In case where opener needs to show some suit feature (shortness, specific K, etc.) but that suit is the asking bid (ie, \di5 for \cl{} Kickback) or would commit to game/slam, the next lowest NT bid is used as the surrogate bid (ie, 5NT to show the K of diamonds).

\chapter{Openings and Continuations}  \label{2}
\section{Overview} \label{2:1}
    HCP notation: [$x$-$y$] indicates HCP range from `$x$' to `$y$' inclusive, with ($x$+) and ($y$-) indicating an upgraded or downgraded hand respectively.  [(14+)-17] is a good 14 to any 17 count. Table of openings is as follows:
    
\begin{center}
    \begin{tabular}{ |c|c|c|c| } 
        \hline
        \cl1                     & [11-21]    & 3+\cl{}s             & open with 3-3 in minors \n
        \di1                     & [11-21]    & 3+\di{}s             & 3-cards only when 4432  \n
        \he1                     & [11-21]    & 5+\he{}s             & could have 6-card minor \n
        \sp1                     & [11-21]    & 5+\sp{}s             & could have 6\he{}s      \n
        1NT                      & [(14+)-17] & (semi)balanced       & could have 5-card major \n
        \cl2                     & [22+]      & or 8.5 tricks        & game-forcing            \n
        2\di{}/\he{}/\sp{}       & [4-10]     & 6-card suit          & preemptive/constructive \n
        2NT                      & [20-21]    & (semi)balanced       & not 9-cards in major    \n
        3\cl{}/\di{}/\he{}/\sp{} & [4-10]     & 7-card suit          & highly preemptive       \n
        3NT                      & [9-11]     & AKQxxxx in minor     & no outside stoppers     \n
        4\cl{}/\di{}             & [7-14]     & 8+ major, 8+ tricks  & transfer up-2           \n
        4\he{}/\sp{}             & [4-10]     & 8+ major, weak suit  & usually fairly weak     \n
        \hline
    \end{tabular}
\end{center}
Contents:
\ul {
    \li0 \ref{2:1} - Overview
    \li0 \ref{2:2} - \cl1
    \li0 \ref{2:3} - \di1
    \li0 \ref{2:4} - \he1
    \li0 \ref{2:5} - \sp1
    \li0 \ref{2:6} - \textit{Gazzilli}
    \li0 \ref{2:7} - 1NT - [(14+)-17]
    \li0 \ref{2:8} - \cl2 - Game-Forcing
    \li0 \ref{2:9} - 2\di{}/\he{}/\sp{}
    \li0 \ref{2:10} - 2NT - [20-21]
    \li0 \ref{2:11} - 3\cl{}/\di{}/\sp{}
    \li0 \ref{2:12} - 3NT - Gambling
    \li0 \ref{2:13} - 4\cl{}/\di{}/\he{}\sp{} - NAMYATS
}
\newpage
\bidsection{\cl{1}}{1♣} \label{2:2}
\begin{center}

    When a major is un-openable, \cl1 is opened with either longer clubs, or 3-3 in the minors. this necessarily puts strain on the minors: \cl1 will have only 3-cards 16\% of the time.  While low, it is not insignificant.\n
    
    Another factor to consider with 1m openings is the 1NT strength.  Since this system uses a [14(+)-17] NT, the weak [12-14] NT hands naturally fall into the 1m openings.  So if a 1m is balanced, it will have either [12-14] or [18-19] HCP.\n

    The most important style used over \cl1 is Walsh, a bypass of diamonds (potentially a 6-card suit) to show a 4-card major suit.  If responder is game-forcing, they shall not bypass diamonds, as they can reverse into the major later.\n

    Conventions (or styles) used include Walsh, inverted minors, splinters, artificial reverses, blackout after reverses (bid cheaper of 4th suit or 2NT), splinters, weak jump shifts, and xyz (excluding \cl1-\di1-\he1 auction, bids are natural with \sp1 as GF). \n

    The full structure is as follows:

    \ul {
        \li0 \di1 - [6+] Walsh style, no major unless GF
        \ul {
            \li0 \he1 - [11-18] 4+\cl{}s, 4+\he{}s, unbalanced
            \li0 \sp1 - [11-18] 4+\cl{}s, 4+\sp{}s, unbalanced (denies hearts)
            \li0 1NT - [12-14] balanced or semi-balanced.  May have 4-card major
            \li0 \cl2 - [11-14] 6+\cl{}s
            \li0 \di2 - [11-14] 4+\di{}s (necessarily 5+\cl{}s)
            \li0 \he2 - [19+] Game forcing.  Either a club single suitor or 4+\he{}s
            \li0 \sp2 - [19+] 4+\cl{}s, 4+\sp{}s.  Game forcing
            \li0 2NT - [18-19] balanced or semi-balanced, Walsh style
            \ul {
                \li0 \cl3 - relay to \di{3}.  After relay:
                \ul {
                    \li0 pass - to play
                    \li0 \he3 - 6-5 or better in \di{}/\he{} (necessarily 12+ by Walsh)
                    \li0 \sp3 - [12+] 6-5 or better in \di{}/\sp{}
                    \li0 3NT - slam try in \cl{}s (opener's minor)
                }
                \li0 \di3 - [12+] 5+ slam try
                \li0 \he3 - [12+] 4\he{}s
                \li0 \sp3 - [12+] 4\sp{}s
            }
            \li0 \cl3 - [15-17] club single suitor.  Typically unbalanced
            \li0 \di3 - [15-17] 4+\di{}s (necessarily 5+\cl{}s)
            \li0 \he3 - [18+] 4+\di{}s, splinter
            \li0 \sp3 - [18+] 4+\di{}s, splinter
        }

        \li0 \he1 - [6+] 4+\he{}s, possible \di{} canapé if weak
        \ul {
            \li0 1NT - [12-14] balanced or semi-balanced.  Use xyz (shown in \di1 structure)
            \li0 \sp1 - [11-18] 4+\sp{}s.  Can be balanced.  Use xyz
            \li0 \cl2 - [11-14] 5+\cl{}s, almost always 6
            \ul {
                \li0 \di2 - artificial GF.  Either 5+\he{}s or slammish in clubs.
                \li0 \he2 - signoff
                \li0 \sp2 - natural responder reverse, GF
                \li0 2NT - [11-12] invite
                \li0 \cl3 - [10-12] invite with support
                \li0 \he3 - [10-12] 6+\he{}s, invite
            }
            \li0 \di2 - [18+] artificial.  Natural \di{} reverse or single suitor
            \ul {
                \li0 \he2 - 5+\he{}s
                \li0 \sp2 - blackout, very weak hand (all other bids GF)
                \ul {
                    \li0 2NT - [18-19] diamond reverse (surrogate)
                    \li0 \cl3 - [18-19] single suitor
                    \li0 \di3 - [20-21] 4-6 or better reverse.  Looking for 5m
                    \li0 3NT - to play
                }
                \li0 2NT - balanced GF
                \li0 \cl3 - 4+\cl{}s
                \li0 \he3 - 6+\he{}s, slammish
            }
            \li0 \he2 - [11-14] 3+\he{}s, almost never 3-card support.
            \ul {
                \li0 Suit - game try, promises 5\he{}s
                \li0 2NT - non-forcing invite
                \ul {
                    \li0 pass - 3\he{}s, deny
                    \li0 \he3 - 4\he{}s, deny
                    \li0 3NT - 3\he{}s, accept
                    \li0 \he4 - 4\he{}s, accept
                }
                \li0 3NT - pass-or-correct
            }
            \li0 \sp2 - [19+] 4+\sp{}s, game forcing
            \li0 2NT - [18-19] balanced or semi-balanced (major system below)
            \ul {
                \li0 \cl3 - transfer to \di{3} for signoff or opener's minor slam try.  After \di3 (passable):
                \ul {
                    \li0 \he3 - 4-5 in majors, pass or correct to \sp3
                    \li0 \sp3 - 4-6 in majors, pass or correct to \he4
                    \li0 3NT - slam try in \cl{}s
                }
                \li0 \di3 - checkback
                \li0 \he3 - 6-card suit, slammish
                \li0 \sp3 - 4-5 in majors
                \li0 Games - to play
            }
            \li0 \cl3 - [15-17] 6+\cl{}s
            \li0 \di3 - [15-17] 4+\he{}s, mini-splinter
            \li0 \he3 - [15-17] 4+\he{}s, unbalanced (by inference: spade shortness)
            \li0 \sp3 - [18+] game-forcing splinter
        }
        \li0 \sp1 - [6+] 4+\sp{}s, denies 4\he{}s unless longer spades
        \ul {
            \li0 \di2 - [18+] artificial.  Either natural diamond reverse or single suitor
            \ul {
                \li0 \he2 - blackout
                \ul {
                    \li0 \sp2 - 3-card support, non-forcing
                }
            }
            \li0 \he2 - [18+] 4+\he{}s 5+\cl{}s, reverse
            \ul {
                \li0 2NT - blackout, relay to \cl3 unless GF
            }
            \li0 \di3 - [15-17] mini-splinter
            \li0 \he3 - [15-17] mini-splinter
        }
        \li0 1NT - [7-10] no major, balanced
        \ul {
            \li0 \cl2 - signoff
            \li0 \di2 - artificial.  Natural reverse or single suited
            \li0 \he2 - natural reverse
            \li0 \sp2 - natural reverse
            \li0 2NT - [15-16] invitational
            \li0 3NT - [17+] signoff
            

        }
        \li0 \cl2 - [10+] 4+\cl{}s, strong raise or [16+] balanced without major
        \ul {
            \li0 \di2 - [12-14] natural
            \li0 \he2 - stopper or awkward GF
            \li0 \sp2 - stopper
            \li0 2NT - [12-14] balanced
            \li0 \cl3 - [12-14] 5+ \cl{}s
            \li0 \di3 - [15+] 5+ \cl{}s, splinter
            \li0 \he3 - [15+] 5+ \cl{}s, splinter
            \li0 \sp3 - [15+] 5+ \cl{}s, splinter
            \li0 3NT - [18-19] balanced
        }

        \li0 \di2 - [0-5] very weak jump
        \li0 \he2 - [0-5] very weak jump
        \li0 \sp2 - [0-5] very weak jump
        \li0 2NT - [11-12] invite
        \li0 \cl3 - [6-9] 5+\cl{}, preemptive.  Correctable with [18-19]
        \li0 \di3 - 5+\cl{}s, splinter (deny majors)
        \li0 \he3 - 5+\cl{}s, splinter (deny major)
        \li0 \sp3 - 5+\cl{}s, splinter (deny majors)
        \li0 3NT - [13-15] balanced
    }

    Passed hands respond identically, but cannot make forcing bids (ie \cl2 is passable).  Since responder cannot have a game force, xyz \di2 shows a maximum invite, generally an [11(+)-12] count that is nearly openable, while \cl2 (still transfer) shows [10-11].

\bidsection{\di{1}}{1♢} \label{2:3}

    \di1 is the simpler of the minor openings.  It promises 3+\di{}s, but only has 3 when holding exactly 4432 shape.  Thus, opener will hold 4+\di{}s 96\% of the time.  Additionally, there is no suit to bypass, so no Walsh treatment is necessary.  As most of the \cl1 structure remains in the \di1 structure, only the differences will be shown:

    \ul {
        \li0 \he1 - 4+\he{}s
        \ul {
            \li0 \sp1 - [11-18] 4+\sp{}s
            \ul {
                \li0 1NT - [7-10] minimum
                \li0 \cl2 - relay to \di2 for any invite or signoff.  Rebids:
                \ul {
                    \li0 \he2 - 5\he{}s
                    \li0 \sp2 - 4\sp{}s and 5\he{}s
                    \li0 2NT - [11-12] natural
                    \li0 \cl3 - [10-12] 5+\clubs{} (typically 6)
                    \li0 \di3 - [10-12] invite with fit
                    \li0 \he3 - [10-12] 6\he{}s
                }

                \li0 \di2 - artificial GF
                \li0 \he2 - [6-10(-)] 5+\he{} (typically 6)
                \li0 \sp2 - natural reverse
                \li0 2NT - transfer to \cl3 (6-card suit)
                \li0 3(their suit) - slam try in suit
                \li0 3(other suit) - 6+ cards, slam try.  3NT denies fit
            }
            \li0 1N - [12-14] balanced
            \ul {
                \li0 \cl2 - relay to \di2 for invite or signoff
                \li0 \di2 - artificial GF
                \li0 \he2 - signoff
                \li0 \sp2 - natural reverse
                \li0 2NT - transfer to \cl3 (6-card suit)
                \li0 3z - flag suit, slam try
            }
            \li0 \sp2 - natural jump shift in \sp{}, or single suitor
            \ul {
                \li0 2NT - blackout
                \li0 \cl3{} - waiting bid (typically 5\he{}s)
                \li0 \di3{} - 3+\di{}s, support
                \li0 \he3{} - 6+\he{}s
            }
            \li0 \cl3 - [19+] 4+\cl{}s, GF
        }
        \li0 \sp1 - 4+\sp{}s, denies 4+\he{}s unless longer spades
        \li0 1NT - [7-10] no major, may have a 6\cl{}s
        \li0 \cl2 - 2/1, 5+\cl{}s.  Does not deny major.
        \li0 \cl3 - [10-12] 6+\cl{}s, good suit quality.  Denies majors
    }

\end{center}

\bidsection{\he{1}}{1♡} \label{2:4}

    Major openings require a 5+ card suit in every seat and vulnerability, requiring [11-21] HCP.  In 3rd seat, one may open with [10] HCP under protection of \cl2 Drury.  \n
    
    Since the higher suit of equal length is opened, \he{1} denies 5\sp{}s, unless opener shows a 5-6 (or better) in the majors. However, with 5-6 in a major-minor, the major is \textbf{always} opened.  Gazzilli allows showing both [14(+)-16] and [17+] of this shape.  This naturally relieves strain on minor-major reverses, which promise exactly a 4-card major (unless artificial).\n

    The most important motif for slam bidding (in any opener) is that as soon as a major fit is discovered, the major is flagged and the partnership is committed to playing in the major, unless both players are balanced (ie 3334 opposite 5332).  Serious cues, non-serious 3NT, Kickback, and EKC are on after a suit is flagged.  \n

    Conventions over 1M include a semi-forcing 1NT, xyz, \textit{Gazzilli}, Jacoby 2NT, Reverse Bergen, balanced 3NT raise, non-serious 3NT, splinters, Kickback (1430), and EKC (0314). Note that jumps to 4M are \textit{always} preemptive as responder has several ways to show support and cuebid/keycard.  A jump to game in the opposite major shows a long (typically self sufficient) suit with [4-8] HCP.  \n

    Because of the complicated nature of \textit{Gazzilli}, \he1-\sp1, \he1-1NT, and \sp1-1NT are in a later section.  Other structure:

\subsection{Game-forces and Raises}
    \ul {
        \li0 \cl2 - 4+\cl{}s, 2/1 GF (elaborated in \sp1 section)
        \li0 \di2 - 4+\di{}s, 2/1 GF
        \li0 \he2 - [8-10] 3+ \he{}s, constructive raise
        \ul {
            \li0 \sp2 - invitational ambiguous splinter.  Relay to 2NT or bid major with min/max.
            \ul {
                \li0 2NT-\he3 - spade shortness (surrogate)
            }
            \li0 2NT - Help suit in spades
            \li0 \cl3 - [15-16] Help suit in clubs
            \li0 \di3 - [15-16] Help suit in diamonds
            \li0 \he3 - [15-16] informationless invite
            \li0 \sp3 - [18+] splinter
            \li0 3NT - pass-or-correct
        }

        \li0 \sp2 - [0-4] 6+ \sp{}s, very weak
        \li0 2NT - 4+\he{}s GF, balanced unless [17+]
        \ul {
            \li0 \cl3 - shortness
            \li0 \di3 - shortness
            \li0 \he3 - [18+]
            \li0 \sp3 - shortness
            \li0 3NT - [15-17] semi-balanced submaximum 
            \li0 4m - side 5-card suit
            \li0 \he4 - [11-14] minimum
        }
        \li0 \cl3 - [10-12] with 4+\he{}s or [12-13] splinter
        \ul {
            \li0 \di3 - artificial asking (invitational or slammish)
            \ul {
                \li0 \he3 - [10], garbage bad hand
                \li0 \sp3 - [12-13] shortness
                \li0 \cl4 - [12-13] shortness
                \li0 \di4 - [12-13] shortness
                \li0 \he4 - [10(+)-12], accept
            }
            \li0 \he3 - to play
            \ul {
                \li0 \he4 - [12-13] ambiguous splinter
            }
        }
        \li0 \di3 - [7-9] with 4+\he{}s, constructive
        \li0 \he3 - [0-6] with 4+\he{}s, preemptive
        \li0 \sp3 - [14-16] splinter (weaker splinters go through \cl3)
        \li0 3NT - [13-16] (4)3(33), pass-or-correct
        \li0 \cl4 - [14-16] splinter
        \li0 \di4 - [14-16] splinter
        \li0 \he4 - [4-9] 5+\he{}s, not balanced unless favorable
        \li0 \sp4 - [4-9], to play.  Flag \sp{}s\n\n
    }

\subsection{Passed Hand Responses}
    Passed hands use natural 1NT/\di2/2NT, \cl2 Drury, and support jump shifts:
    \ul {
        \li0 1NT - [6-10] no fit
        \li0 \cl2 - [10-11] 3+\he{}s.  Drury, denies a support jump shift
        \ul {
            \li0 \di2 - [13], looking for a maximum.  Could be slammish.
            \li0 \he2 - Signoff
        }
        \li0 \di2 - [10-11] 5+\di{}s.  Denies a fit
        \li0 \he2 - [6-9] 3+\he{}s.
        \li0 \sp2 - [10-11] 5+\sp{}s, 4\he{}s.
        \li0 2NT - [11-12] no fit, invite.
    }

\bidsection{\sp{1}}{1♠} \label{2:5}

    Unlike \he1, \sp1 has a simpler structure since responder either shows a fit with defined strength through a myriad of raises, or immediately limits their hand via semi-forcing 1NT. Passed hand responses are identical. \n

    As mentioned, \textit{Gazzilli} responses are located in \ref{2:6}.  The structure is as follows:

    \ul {
        \li0 \cl2 - 3+\cl{}s, 2/1 GF (3 only when 3433)
        \ul {
            \li0 \di2 - [11-21] 4+\di{}s (lower suits are free to show)
            \li0 \he2 - [11-21] 4+\he{}s
            \li0 \sp2 - [11-15] Default response
            \li0 2NT - [14] or [18-19] balanced (show slammish strength later)
            \li0 \cl3 - [14+], 4+\cl{}s (raises show 16+ playing strength).  Flagged unless major is \textit{immediately} supported
            \li0 \di3 - [16+], 5-5 in \di{}/\sp{}s with great suit quality
            \li0 \he3 - [16+], 5-5 in \he{}/\sp{} with great suit quality
            \li0 \sp3 - [18+], 6+\sp{}s single suited.  Flagged
        }
        \li0 \di2 - 4+\di{}s, 2/1 GF
        \ul {
            \li0 \cl3 - [16+] 4+\cl{}s, typically unbalanced (reverses show HCP)
        }
        \li0 \he2 - 5+\he{}s, 2/1 GF
        \ul {
            \li0 \he3 - [14+] 3+\he{}s (slow shows).  Flags hearts
            \li0 \he4 - [11-13] 3+\he{}s (fast arrival, typically balanced)
        }
        \li0 \sp2 - [8-10] constructive raise
        \ul {
            \li0 2NT - invitational ambiguous splinter.  Relay to \cl3 or bid major with min/max.
            \ul {
                \li0 \cl3- club shortness (surrogate)
            }
        }
        \li0 \he3 - [14-16] 4+\sp{}s, splinter
        \li0 \he4 - 7+\he{}s, preemptive jump
    }

    For 2/1 auctions, jumps above reverses are splinters such as: \he1-\cl2-\sp3.  In case of a splinter in minor, you sacrifice Kickback, which is accessible by setting trumps via a natural raise (\sp1-\di2-\he4 is a splinter, since \di3 confirms a fit for Kickback)

\section{Gazzilli} \label{2:6}

    A few notes for \textit{Gazzilli}: the bids are designed to be as natural as possible: jumps show good shape and other major initiates an shape-showing relay.  Whether the opener goes through \cl2 or not, the structure is similar.  If \cl2 is doubled, XX shows [8+], \di2 is weak with diamonds, and pass shows other weak hands.  If \di2 is doubled, pass shows the weak hand, redouble shows strong hand with no direction, otherwise system on. \n

    The \textit{Gazzilli} auctions are:

    \ul {
        \li0 \he1-\sp1 - 4+\sp{}s, denies 4\he{}s, may have 3-card support.  Forcing \ul {
            \li0 1NT - [12-14] minimum, balanced or semi-balanced.  Denies 4\sp{}s.  Use xyz
            \li0 \cl2 - [11-16] with 4+\cl{}s or any [17+] (\textit{Gazzilli})
                \li1 \di2 - [8+] any, GF opposite strong hand
                    \li2 \he2 - [11-16] with 4+\cl{}s.  Respond as if \cl2 was bid
                    \li2 \sp2 - natural 4+\sp{}s OR unbalanced 4-card minor.  Relay to 2NT or \he3
                        \li3 2NT - relay
                            \li4 \cl3 - 4\cl{}s exactly
                                \li5 \di3 - 1-\he{} cue for clubs
                                \li5 \he3 - 2\he{}s
                                    \li6 cue - 6+\he{}s, interest
                                    \li6 3NT - 5\he{}s
                                        \li7 \cl4 - slammish in clubs
                                \li5 3NT - no fits, to play
                            \li4 \di3 - 4\di{}s exactly
                            \li4 \he3 - 4-6 in majors (\sp2-\he3 were natural)
                            \li4 \sp3 - 5-6 in majors (\sp2-\sp3 were natural)
                        \li3 \he3 - 3-card support.  Flag suit
                    \li2 2NT - [17-19] balanced or semi-balanced
                    \li2 \cl3 - big 5-(5+) in \cl{}/\he{}
                    \li2 \di3 - big 5-(5+) in \di{}/\he{}
                    \li2 \he3 - 6+\he{}s
                    \li2 \sp3 - ***
                    \li2 3NT - ***
                    \li2 \cl4 - big 6-(5+) in \cl{}/\he{}
                    \li2 \di4 - big 6-(5+) in \di{}/\he{}
                \li1 \he2 - [6-7] 2-3\he{}s
                \li1 \sp2 - [6-7] 1-\he{}s, 5+\sp{}s
                \li1 2NT - minors, typically longer diamonds
                \li1 \cl3 - [6-7] 5+\cl{}s (typically 6)
                \li1 \di3 - [6-7] 5+\di{}s (typically 6)
            \li0 \di2 - [11-16] with 4+\di{}s, unbalanced
                \li1 \he2 - signoff
                \li1 \sp2 - signoff
                \li1 2NT - invite
                \li1 \cl3 - 4th suit forcing
                \li1 \di3 - [9-11] 4+ \di{}s, invite to game
                \li1 \he3 - [11-12] 3-card limit raise
                \li1 \sp3 - [9-11] 6+\sp{}s
            \li0 \he2 - [11-14] with 6+\he{}s
            \li0 \sp2 - [11-16] with 4\sp{}s
            \li0 2NT - [14(+)-16] with 4\sp{}s, ambiguous splinter
                \li1 \cl3 - singleton ask
                    \li2 \di3 - in \di{}s
                    \li2 \he3 - in \cl{}s (surrogacy principle)
                \li1 3\di{}/\he{} - cue
                \li1 \sp3 - to play
            \li0 \cl3 - [14(+)-16] with 5-5 in \cl{}/\he{}
                \li1 \di3 - flag \cl{}s (lower cue for lower suit)
                \li1 \he3 - to play
                \li1 \sp3 - flag \he{}s
                \li1 3NT - to play
            \li0 \di3 - [14(+)-16] with 5-5 in \di{}/\he{}
            \li0 \he3 - [14(+)-16] with 7+\he{}s.  Flag suit
                \li1 \sp3 - cue
                \li1 3NT - to play
            \li0 \sp3 - [14(+)-16], 4522.  Flag \sp{}s
            \li0 \cl4 - [14(+)-16] 6-(5+) in \cl{}/\he{}
            \li0 \di4 - [14(+)-16] 6-(5+) in \di{}/\he{}
        }
        \li0 \he1-1NT - invitational-, denies 4-card major.  Non-forcing \ul {
            \li0 \cl2 - \textit{Gazzilli}
                \li1 \di2 - [8+] any
                    \li2 \he2 - [11-16] 3+\cl{}s
                    \li2 \sp2 - either natural or exactly 4-card minor.  Relay to 2NT
                    \li2 2NT - [17-19] balanced or semi-balanced
                    \li2 \cl3 - 5-(5+) in \cl{}/\he{}
                    \li2 \di3 - 5-(5+) in \di{}/\he{}
                    \li2 \he3 - 6+\he{}s
                    \li2 \sp3 - ***
                    \li2 3NT - ***
                    \li2 \cl4 - 6-(5+) in \cl{}/\he{}
                    \li2 \di4 - 6-(5+) in \di{}/\he{}
                \li1 \sp2 - [6-7] (4-4) or better in minors, equal or longer clubs
            \li0 \di2 - 3+ \di{}s ([14+] 4531 and 3532)
            \li0 \sp2 - [11-14] 5\sp{}s, 6\he{}s (weak 5-6 in majors)
            \li0 2NT - *** (shows unbalanced raise in \he1-\sp1, what here?)
        }
        \li0 \sp1-1NT - invitational-, denies 4\sp{}s.  Non-forcing \ul {
            \li0 \cl2 - Gazzilli
                \li1 \di2 - [8+]
                    \li2 \he2 - natural or unbalanced 4-card suit.  Relay
                        \li3 2NT - 5422
                        \li3 \cl3 - 4-(5+) in \cl{}/\sp{}
                        \li3 \di3 - 4-(5+) in \di{}/\sp{}
                        \li3 \he3 - unbalanced 4-5 in \he{}/\sp{}
                        \li3 \sp3 - 4-6 in \he{}/\sp{}s
                \li1 \he2 - [6-7] 5+\he{}s, 1-\sp{}
                \li1 \sp2 - [6-7] 2+\sp{}s
                \li1 2NT - [6-7] 4-4 or better in minors, typically longer diamonds
                \cl3 - [6-7] 5+\cl{}s
                \di3 - [6-7] 5+\di{}s
            \li0 \di2 - [11-16] 4+\di{}s
            \li0 \he2 - [11-16] 4+\he{}s
            \li0 \sp2 - [11-14] 6+\sp{}s
            \li0 2NT - [14(+)-16] 6-4 in the majors
            \li0 3z - [14(+)-16] 5-(5+) in z/\sp{}s
            \li0 \sp3 - [14(+)-16] 7+\sp{}s, single suited
            \li0 4z - [14(+)-16] 6-(5+) in z/\sp{}s (strong 5-6 in majors)
        }
    }

    A consequence of \he1-1NT-\sp2 and \sp1-1NT-\he4 both being limited bids (but showing a 5-6 in majors), it is natural to define the reverse as weaker (as \sp2 and \he3 are escapes) and the jump shift as strong.  Thus, \sp1 may be a canape with hearts as a longer suit!
        

\bidsection{1NT [(14+)-17]}{1NT [(14+)-17]} \label{2:7}

    Any 15-17 with (4333), (4432), or (5332) distribution are opened 1NT.  The semi-balanced hands (5422), (6322) may be opened 1NT only if the longest suit is a minor. 14 HCP hands with a strong 5+ suit may be upgraded to 1NT.  When responder range-asks, a minimum is [14(+)-15], and maximum are [16-17].  A 15-count with good controls and shape may be upgraded \textit{only} when responder is slammish.
    \\

    The main feature of this NT structure is exploring responder's shape while efficiently using bidding space.  As bridge is about the majors, most conventions are geared toward identifying 4-4 and 5-3 major fits.\n

    However, when responder and opener have a good minor fit with shape/points, it is desireable to be in a safer minor game or slam slam as opposed to 3NT or 6NT.  Knowing the size of a fit is important, so responder's first bids show their suit length. For slammish hands, this is the following structure: \n
    
    With a balanced hand without a major, responder can use \sp2 or \cl3 puppet followed by a quantitative jump to 4NT, after which the 4NT sequence is used (see below). \n
    
    For 6-card slammish minor hands, responder can transfer, splinter, and keycard.  With a 4-card major, use puppet and bid the minor.  Minors after puppet promise 6 cards. \n

    With unbalanced 5-card minor hands without a major, responder has \di3, \he3, and \sp3.  Thus, the last class of hands are 5431 hands with a 5-card minor and a 4-card major.  These are shown through \cl2 stayman, with a conventional sequence to show responders exact shape. \n\n


    The full structure is as follows:
    \ul{
        \li0 \cl2 - stayman (not 5-5 in majors) or slammish with \textit{exactly} 5 cards in a minor
        \ul{
            \li0 \di2 - no majors; South African transfers ON
            \ul{
                \li0 2M - 5-card invite
                \li0 2NT - invite
                \li0 \cl3 - slammish with 5\cl{}s
                \ul{
                    \li0 Step 1 - minimum, fit
                    \li0 suits - max, cue
                    \li0 3NT - no fits
                    \li0 \cl4 - max, cue
                    \li0 \di4 - Kickback
                }

                \li0 \di3 - slammish with 5\di{}s. Use above continuation

                \li0 3M - 5 in other major, GF (smolen)
            }
            \li0 \he2 - 4+\he{}s
            \ul {
                \li0 \sp2 - 5\sp{}s 4\he{}s invite
                \li0 2NT - 4\sp{}s invite
                \li0 \cl3 - 3-card raise, 4 in other major.  Slammish with (1-5) in minors
                \ul {
                    \li0 \di3 - specify shape
                        \li0 Step 1 - 5\cl{}s, slam invite
                        \li0 Step 2 - 5\di{}s, slam invite (may be 3NT, passable)
                        \li0 Step 3 - 5\cl{}s, slam forcing (may be 3NT, not passable)
                        \li0 Step 4 - 5\di{}s, slam forcing

                    \li0 3M - fit, slam interest
                    \li0 4M - fit, no interest
                }
                \li0 \di3 - singleton or void in major.  40(45) or 41(35) shape.  Use above continuation
                \li0 \he3 - invite
                \li0 \sp3 - slammish \he{} raise
                \li0 3NT - 4\sp{}s, pass or correct.
                \li0 4NT - quantitative
                \li0 5NT - pick a slam (must have 4\sp{}s)
            }

            \li0 \sp2 - 4+\sp{}s, denies 4\he{}s.  Use above continuation
        }
        \li0 \di2 - 5+\he{} transfer
        \ul {
            \li0 \he2 - accept
            \ul {
                \li0 \sp2 - 5-5 majors invite
                \li0 2NT - 5\he{}s invite
                \li0 \cl3 - 4+\cl{}s GF
                \ul {
                    \li0 \di3 - 4+\cl{}s, minimum, denies heart fit
                    \li0 \he3 - 3+\he{}s, with extras
                    \li0 \sp3 - 4+\cl{}s, maximum, denies heart fit
                    \li0 3NT - no fit
                }
                \li0 \di3 - 4+\di{}s GF.  Use above continuation
                \ul {
                    \li0 \cl4 - 4+\di{}s, maximum
                }
                
                \li0 \he3 - 6+\he{}s invite
                \li0 \sp3 - splinter
                \li0 3NT - pass or correct
                \li0 \cl4 - splinter
                \li0 \di4 - splinter
                \li0 \he4 - 6332 slam invite
                \li0 \sp4 - EKC 0314
                \li0 4NT - quantitative
            }
            \li0 \he3 - superaccept; 5\he{}s or [16-17] 4\he{}s, not 4333
            \ul {
                \li0 cues, Kickback, EKC on
            }
        }

        \li0 \he2 - 5+\sp{} transfer. Use above continuation from \di2
        \ul {
            \li0 \sp2 - accept
            \ul {
                \li0 \he3 - 5-5 majors GF
            }
        }

        \li0 \sp2 - 6+\cl{} weak, 6+\cl{} GF, or range-ask
        \ul {
            \li0 2NT - [15]
            \ul {
                \li0 \cl3 - signoff
                \li0 \di3 - splinter
                \li0 \he3 - splinter
                \li0 \sp3 - splinter
                \li0 3NT - to play
                \li0 \cl4 - good slam invite
                \li0 \di4 - Kickback
            }
            \li0 \cl3 - [16-17]. Use above continuation
        }

        \li0 2NT - 6+\di{} weak, 6+\di{} GF, or a very weak 5-5 in \cl{}/\di{}
        \ul {
            \li0 \cl3 - 2-\di{}s or 3\di{}s and 5+\cl{}s.  With 2245 bid \di3 as opponents probably have game
            \ul {
                \li0 \di3 - signoff
                \li0 \he3 - splinter
                \li0 \sp3 - splinter
                \li0 3NT - mild slam invite, passable
                \li0 \cl4 - splinter
                \li0 \di4 - good slam invite
                \li0 \he4 - Kickback
            }
            \li0 \di3 - 3+\di{}s, any other hand. Use above continuation
        }

        \li0 \cl3 - puppet stayman, or slammish 6+minor with 4-card major.

        \ul {
            \li0 \di3 - no 5-card major, does \textit{not} promise a major
            \ul {
                \li0 \he3 - 4\sp{}s
                \li0 \sp3 - 4\he{}s
            }
            \li0 \he3 - 5\he{}s
            \ul {
                \li0 \sp3 - forcing \he{} raise.
                \li0 3NT - to play
                \li0 4m - 6+ suit, natural slam invite.  Typically 4\sp{}s.
            }
            \li0 \sp3 - 5\sp{}s
        }
        \li0 \di3 - 5-5 or better in \cl{}/\di{} GF
        \ul {
            \li0 \he3 - flag clubs, cooperative
            \li0 \sp3 - flag diamonds, cooperative
            \li0 3NT - double stops in both majors 44(32) or (53)(32)
            \li0 4m - minimum, support for minor, typically with major wastage
        }
        \li0 \he3 - 31(45) GF.  Singleton not A/K unless slammish
        \ul {
            \li0 \sp3 - 4\sp{}s, looking for 4-3 fit
            \li0 3NT - double stopper
            \li0 \cl4 - good clubs (or 33)
            \li0 \di4 - good diamonds
            \li0 \he4 - maximum, double minor fit
            \li0 \sp4 - 5\sp{}s, to play
        }
        \li0 \sp3 - 13(45) GF
        \li0 3NT - to play
        \li0 \cl4 - 6+\he{}; transfer to play, 1430, or cuebid
        \li0 \di4 - 6+\sp{}; transfer to play, 1430, or cuebid
        \li0 \he4 - to play
        \li0 \sp4 - to play
        \li0 4NT - artificial, explained below
    }

    Since \sp2 is an immediate range ask, 4NT is not needed as a natural invite to 6NT.  Instead, it shows a slam-forcing balanced hand with some interest in a minor slam.  Opener's responses are mostly natural:

    \ul {
        \li0 \cl5 - 4+\cl{}s, shorter diamonds
        \li0 \di5 - 4+\di{}s, shorter clubs
        \li0 5NT - 4-4 in minors
        \li0 \cl6 - 6\cl{}s
        \li0 \di6 - 6\di{}s
        \li0 6NT - no interest
    }

    Note that this convention \textit{only} applies after 1NT and 1NT-\sp2-Z.  4NT is needed as natural (or \sp{} kickback) in other scenarios.
    
\bidsection{\cl{2} - Game-Forcing}{2♣ - Game-Forcing} \label{2:8}

    \cl2 is the strongest opening bid, showing a game forcing hand or [22-24] balanced (which should almost always be raised to game anyway).  \cl2 has waiting and positive responses, Kokish \he2, cheaper minor, \di3 stayman, and conventional 3M jump rebids by opener.\n
    
    The structure is as follows:

    \ul {
        \li0 \di2 - no positive response, a positive NT, or a positive diamond with a 4-card major.
        \ul {
            \li0 \he2 - 5+\he{}s or [25+] balanced.  Forced relay to \sp2.
            \li0 \sp2 - 5+\sp{}s
            \li0 2NT - [22-24] balanced
            \li0 \cl3 - 5+\cl{}s
            \ul {
                \li0 \di3 - stayman
            }
            \li0 \di3 - 6+\di{}s.  No 4-card major
            \li0 \he3 - 4\he{}s, 5+\di{}s
            \ul {
                \li0 \sp3 - forcing heart raise
                \li0 3NT - to play
                \li0 \cl4 - cue for diamonds (denies hearts)
                \li0 \di4 - cue for diamonds (denies hearts)
                \li0 \he4 - to play
            }
            \li0 \sp3 - 4\sp{}s, 5+\di{}s
            \ul {
                \li0 \he4 - forcing spade raise (just like stayman, use other major)
            }
        }
        \li0 \he2 - 5+\he{}s, KQ or better
        \ul {
            \li0 \sp2 - 5+\sp{}s
            \li0 2NT - [22-24] balanced.  Bid naturally
            \li0 \cl3 - 5+\cl{}s
            \li0 \di3 - 5+\di{}s
            \li0 \he3 - Flag suit.
        }

        \li0 \sp2 - 5+\sp{}s, KQ or better.  Use \he2 structure.
        \li0 \cl3 - 5+\cl{}s, KQ or better
        \ul {
            \li0 \di3 - stayman, or \di{} single suitor
            \li0 \he3 - 5+\he{}s
            \li0 \sp3 - 5+\sp{}s
            \li0 3NT - [22] dead minimum, no fit
            \li0 \cl4 - 3+ \cl{}s, fit.  Cues preferable over keycard
            \li0 \di4 - Kickback
        }
        \li0 \di3 - 6+\di{}s, KQ or better.  5-card suit acceptable with 3-3 in majors.  Use \cl3 structure.
    }

\bidsection{2\di{}/\he{}/\sp{}}{2♢/♡/♠} \label{2:9}

    Preemptive 2-level openings jam the auction for the opponents, while also giving a vivid description of opener's hand.  With \cl2 able to absorb all strong hands, the other 2-level suit bids are freed up to show weak hands with a long suit. \n

    The first of the weak-2 opening bids, \di2 is perhaps the most interesting.  It eats up bidding space while forcing the opponents to consider both majors.  \he2/\sp2 have the benefit of more often reaching \he4/\sp4, which makes it easier to reach a safe game for the partnership.  Regardless of suit, it is important for the opener to have a well-defined hand for when their partner is preempted.  Since preempts depend on both vulnerability and seat, the strength and quality definitions are as follows:

    \ul {
        \li0 non-vulnerable
        \ul {
            \li0 1st seat - [4-10], JTxxxx or better
            \li0 2nd seat - [6-10], requires a good feature of the hand
            \li0 3rd seat - [4-10], (no restriction on quality/shape)
            \li0 4th seat - [10-14], worse than \di1-\di2
        }
        \li0 vulnerable
        \ul {
            \li0 1st/3rd seat - [5-10], QJT or better
            \li0 2nd seat - [8-10], very nice preempt
            \li0 4th seat - [10-14], worse than \di1-\di2
        }
    }

    When opening a preempt, one must consider their major holding.  Opposite an unpassed partner, a preempt should not contain another 4-card suit, especially a major (exceptions being both minors or very weak).  When opening a major, it is a liability to hold 3-card support for the other major to avoid missing 5-3 fits.  When responder has a fit and wishes to preempt further, they may elect to make a non-forcing raise (RONF).  All other actions are forcing and invitational+.\n
    
    After responder bids a new suit, opener should retreat to their suit with a [4-6] non-fit, raise once with a [4-6] and 3-card support, raise twice with [7-10] 3-card fit, and cue a feature otherwise.

    \ul {
        \li0 \sp2 - [17+] 5+\sp{}s
        \ul {
            \li0 \cl3 - [7-10] feature, no fit
            \li0 \di3 - [7-10] feature, no fit
            \li0 \he3 - [4-6] minimum, no fit
            \li0 \sp3 - [4-6] minimum, fit
            \li0 \sp4 - [7-10] maximum, fit
        }
        \li0 2NT - OGUST
        \ul {
            \li0 \cl3 - [4-7] JTxxxx or worse (1-/3)
            \li0 \di3 - [4-7] Qxxxxx or better (2/3)
            \li0 \sp3 - [8-10] Qxxxxx or worse (1-/3)
            \li0 \sp3 - [8-10] KQxxxx or better (2/3)
            \li0 3NT - AKQxxx (3/3)
        }
        \li0 \cl3 - [17+] 5+\cl{}s
        \ul {
            \li0 \he3 - [4-6] minimum
            \li0 \sp3 - [7-10] no fit.  Note opener cannot bypass 3NT
            \li0 \cl4 - [4-6] fit
            \li0 \cl5 - [7-10] fit
        }
        \li0 \di3 - [17+] 5+\di{}s
        \li0 \he3 - [0+] with support (preemptive)
        \li0 3NT - to play
        \li0 \he4 - [0+] with support (preemptive or strong)
    }

    Note that point ranges are rough guidance. \hhand{KT9, x, QJT9xxx, xxx} is worth much more than [6] opposite a \sp2 response, while \hhand{xx, QJ, KJxxxx, QJx} is worth less than the [9] advertised.  New suits after OGUST are cuebids for the preempted suit, game-forcing and possible slam interest.

\bidsection{2NT [20-21]}{2NT [20-21]} \label{2:10}

    2NT may be opened with shapes (4333), (4432), (5332), (5422), (6332), and (5431).  Singletons must be either A or K, opener must not have a 6-card major or 5-4 in the majors.\n

    Puppet and transfers are used to investigate major fits, and \sp3 is used for hands with minors.  Single suited minors go through puppet ***; this leaks information but is easy on memory. The full structure is as follows:

    %Since opener has more points than responder, 4M is \textbf{not} to play, and instead shows a six card minor (two-below) with slam interest (inspired from Scanian methods).

    \ul {
        \li0 \cl3 - puppet stayman; minors show a 5+ (usually 6) card suit
        \ul {
            \li0 \di3 - at least one 4-card major
            \ul {
                \li0 \cl4 - natural, clubs
                \li0 \di4 - natural, diamonds
                \li0 \he4 - pass-or-correct ***

            }
        }
        \li0 \di3 - 5+\he{} transfer (with EKC jumps)
        \ul {
            \li0 \he3 - accept
            \ul {
                \li0 \sp3 - ***
                \li0 3NT - pass or correct
                \li0 \cl4 - 4+\cl{}s
                \ul {
                    \li0 \di4 - 4+\cl{}s, denies \he{} support
                    \li0 \he4 - 3+\he{}s
                    \li0 \sp4 - ***
                    \li0 4NT - no support
                }
                \li0 \di4 - 4+\di{}s
                \li0 \he4 - mild heart slam try, typically (6331)
                \li0 \sp4/5m - EKC
                \li0 4NT - quantitative

            }
            \li0 \he4 - 4+\he{}s, superaccept (good outside tricks)
        }
        \li0 \he3 - 5+\sp{} transfer.  Use above continuation
        \li0 \sp3 - slammish with (5-4)+ in minors
        \ul {
            \li0 3NT - bad slam hand
            \ul {
                \li0 \cl4 - 5-5 in minors
                \ul {
                    \li0 \di4 - Kickback for \cl{}
                    \li0 \he4 - Kickback for \di{}
                    \li0 \sp4 - cue (typically double fit)
                    \li0 4NT - double negative
                    \li0 5m - fit, 2 keycards
                }
                \li0 \di4 - Kickback for \cl{}
                \li0 \he4 - Kickback for \di{}
                \li0 \sp4 - ***
            }
            \li0 4m - 4+ cards, flag suit
            \li0 4M - ***
        }
        \li0 3NT - to play
        \li0 4m - transfer to \he4/\sp4, to play, cuebid, or 1430
        \li0 4M - ***
        \li0 4NT - [12], quantitative
        \li0 5m - to play
    }
\bidsection{3\cl{}/\di{}/\he{}/\sp{}}{3♣/♢/♡/♠} \label{2:11}

    Unlike 2-level preempts, 3-level preempts are mostly non-constructive.  They almost always show a 7-card suit; the exceptions occur 3rd seat at favorable vulnerability or \cl3 (since \cl2 is not preemptive).  Since 4m is artificial, 3-level minor preempts may have an 8-card suit, though 8-card suits may also be opened at the 5-level. \n

    If opener preempted \cl3 or \di3, majors are natural and forcing (typically 6-cards).  Barring 3m-3M, suits are cues/Kickback/splinters, slam invitational or better.

\bidsection{3NT - Gambling}{3NT - Gambling} \label{2:12}

    3NT shows a 7-card (or better) minor headed by AKQ (or better).  In 1st/2nd seat, opener may have no more than an outside Q.  In 3rd/4th seat, opener promises \textit{exactly} one outside stopper. Responder may correct to any level of minor (pass or correct) or cue a major.

\bidsection{4\cl{}/\di{}/\he{}/\sp{} - NAMYATS}{4♣/♢/♡/♠ - NAMYATS} \label{2:13}

    The 4-level openings are reserved for 8-card major preempts.  Direct 4M is weak, showing [4-10] HCP and an 8-card suit, typically broken unless weak on the outside suits.\n

    \cl4 and \di4 are transfers to \he4 and \sp4 respectively.  These show a self-sufficient suit (3/4 honors; 7+ tricks) and outside values (8 tricks total).\n

    Accepting the transfer is signoff.  Bidding the intermediate suit is an artificial asking bid.  Retreat to 4M shows a minimum 8 tricks, any further bid is a cuebid.\n

    Kickback is on by responder; cues on for both.  The schema is as follows:

    \ul {
        \li0 \cl4 - NAMYATS, 8-card or better heart suit and 8+ tricks
        \ul {
            \li0 \di4 - Artificial asking bid
            \ul {
                \li0 \he4 - minimum 8 tricks
                \li0 \sp4 - spade cue, 8.5+ tricks
            }
            \li0 \sp4 - Kickback
            \li0 4NT - spade cue
            \li0 \cl5 - club cue
        }
        \li0 \he4 - Weak preempt, 8-card broken heart suit or less than 8 tricks.  Use above continuation
    }

\chapter{Interference}  \label{3}
\section{Overview} \label{3:1}

Unfortunately, bridge is not a simple as a back-and-forth conversation with our partner.  Thus, we must have methods for competitive auctions, both when our opponents preempt and overcall.  Since both opener's and overcaller's hands have a suit, our methods show the probable and important hands of responder.  Bids in parenthesis are interference: \cl1-(\he1)-X-(XX). \n

The following conventions are used:
\begin{itemize} \itemsep0em
    \item Takeout, negative, support, optional, reopening X
    \item Penalty, support, value XX
    \item Transfers over 1M-(X)
    \item Jordan 2NT
    \item lebensohl
    \item `Lower-lower' cues
\end{itemize}
Contents:
\ul {
    \li0 \ref{3:1} - Overview
    \li0 \ref{3:2} - Suit-based Interference - 2nd Seat
    \li0 \ref{3:3} - Suit-based Interference - 4th Seat
    \li0 \ref{3:4} - NT Interference
    \li0 \ref{3:5} - Preempts
    \li0 \ref{3:6} - Two-Suited Overcalls
    \li0 \ref{3:7} - Slam Interference
}
\newpage

\section{Suit-based Interference - 2nd Seat} \label{3:2}
    
    Over an opposing double, responder may pass with [0-6] points and no fit, as a strong opener may clarify a strong hand later.  With [7+], responder owes a response.  XX shows balanced [10+], without a 9-card fit.  If a minor was opened, new suits are natural freebids, and 1NT is natural; if a major was opened, 1NT-2(M-1) are transfers, with 2M as a preemptive raise.\n

    Additionally, jump shifts show a quality 5-card suit with 4-card support and constructive values.  2NT shows a 9-card fit and invitational+ values (Jordan 2NT).  Suits above a jump shift are game-forcing splinters.  3NT is natural, typically showing majors stopped. \n

    Over a suit overcall, X is negative, and freebids are natural invitational+, always forcing.  Cues show invitational+ with support.  Jumps are preemptive.\n

    Here is the common auctions against direct suit-based overcalls:
    
    \ul {
        \li0 \cl1-(X)
        \ul {
            \li0 pass - [0-6]
            \li0 XX - [10+] no 9-card fit.  Typically a natural 2NT response.  Penalty X on
            \li0 1z - [7+] 4-card suit or better (system on)
            \li0 1NT - [7-9] balanced, no majors
            \li0 \cl2 - [7-9] 4+\cl{}s
            \li0 \di2 - [7-10] quality 5+\di{}s, 4\cl{}s
            \li0 \he2 - [7-10] quality 5+\he{}s, 4\cl{}s
            \li0 \sp2 - [7-10] quality 5+\sp{}s, 4\cl{}s
            \li0 2NT - [11+] 5+\cl{}s
            \li0 \cl3 - [4-7] 4+\cl{}s, preemptive
            \li0 3z - 5+\cl{}s, splinter.
        }

        \i \cl1-(\di1)
        \ul {
            \li0 pass - [0-6] or [10+] with interest in penalizing \di1
            \li0 X - \textit{exactly} 4-4 in \he{}/\sp{}
            \li0 \he1 - [7+] 4+\he{}s, system on
            \li0 \sp1 - [7+] 4+\sp{}s, system on
            \li0 1NT - [7-10] typically a stopper
            \li0 \cl2 - [7-10] 4+\cl{}s
            \li0 \di2 - [11+] 4+\cl{}s
            \li0 2M - [4-7] 6-card suit, preemptive
            \li0 2N - [11-12] with a stopper
            \li0 \cl3 - [4-6] 5+\cl{}s, weak
            \li0 3M - [4-7] 7-card suit, preemptive
        }

        \li0 \cl1-(\he1)
        \ul {
            \li0 X - [7+] \textit{exactly} 4\sp{}s, or GF balanced without a stopper (cue later)
            \li0 \sp1 - [7+] 5+\sp{}s
            \li0 \di2 - [11+] 5+\di{}s
        }

        \li0 \cl1-(\sp1)
        \ul {
            \li0 X - [7+] 4\he{}s, or [7-10], 5+\he{}s (too weak for a freebid)
        }

        \li0 \he1-(X)
        \ul {
            \li0 XX - [10+] may have 3-card support
            \li0 \sp1 - [7+] 4+\sp{}s
            \li0 1NT - [4-7] 6+\cl{}s or [8-9] 5+\cl{}s
            \ul {
                \li0 \cl2 - [11-17] 1+\cl{}
                \ul {
                    \li0 pass - [4-7] 6+\cl{}s
                    \li0 \he2 - [8-9] 2\he{}s, 5+\cl{}s
                    \li0 \cl3 - [8-9] 6+\cl{}s, quality suit
                }
                \li0 \di2 - [11-17] 4+\di{}s, void in clubs
                \li0 \he2 - [11-17] 6+\he{}s
                \li0 \sp2 - [18+] artificial.  Natural reverse, 6+\he{}s GF, or 3+\cl{}s GF.
                \li0 2NT - [18-19] balanced
                \ul {
                    \li0 \cl3 - [4-5] 6+\cl{}s, signoff
                }
                \li0 \cl3 - [14-17] 3+\cl{}s
                \li0 \di3 - [19+] 4+\di{}s, game-forcing
            }
            \li0 \cl2 - [4-7] 6+\di{}s or [8-9] 5+\di{}s
            \li0 \di2 - [7-10] 3\he{}s, constructive
            \li0 \he2 - [0-6] 3\he{}s, preemptive raise
        }
    }

    Often, responder is too weak, balanced, or strong in the overcalled suit to respond.  Thus, opener should try to reopen with X with almost any hand, as it would be distastrous for our opponents to play \sp1-4 for -400 instead of -1100 when responder holds \spades{AQT872} over overcaller has \spades{KJ965}. \n
    
    With strong hands, simply make the reverse bid or NT cheaply to show strength.  Unfortunately, advancer may raise.  Again, all actions by opener are strong, but a weak responder needs an escape via lebensohl.  Some examples:

    \ul {
        \li0 \cl1-(\sp1)-p-(p)
        \ul {
            \li0 X - [12+] short in spades (almost all hands)
            \li0 1NT - [18-19] balanced with stopper
            \li0 \di2 - [18+] natural reverse
            \li0 \he2 - [18+] natural reverse
        }
        \li0 \di1-(\he1)-p-(\he2)
        \ul {
            \li0 X - [17+] takeout
            \ul {
                \li0 \sp2 - signoff
                \li0 2NT - lebensohl, puppet to \cl3 for signoff
                \li0 3z - [6-8], invitational
            }
        }
    }
\section{Suit-based Interference - 4th Seat} \label{3:3}

    When the 4th suit overcalls a suit after in auction x-y-(z), opener already has shown [11+] strength and responder is able to bid.  Since all bids (including pass!) show [11+], bids can be more descriptive.\n

    Importantly, we have access to support X (or XX), showing \textit{exactly} 3-card support for responder's suit.  This is on through (z)=(\he2), as the partnership must be able to stop at the 2-level in a 4-3 fit.  For \sp2 and higher, double by opener are strong, typically denying support and penalty-oriented.  Some sample sequences (identical bids are ignored, \cl1-\he1-(\sp1)-\di3 is still a mini-splinter):

    \ul {
        \li0 \cl1-\he1-(\sp1)-
        \ul {
            \li0 X - [11-21] \textit{exactly} 3-card support.  \textbf{All other bids deny 3-card support}
            \li0 pass - no other bid, typically [12-14] balanced, no stopper
            \li0 1NT - [12-14] balanced with spade stopper
            \li0 \sp2 - [18-19] balanced without a stopper
            \li0 2NT - [18-19] balanced with a stopper
        }

        \li0 \di1-\sp1-(\he2)-
        \ul {
            \li0 X - support double
            \li0 pass - no other bid
            \li0 \sp2 - [11-14] 4\sp{}s
            \li0 2NT - [18-19] balanced with a stopper
            \li0 \he3 - [18-19] without a stopper (pushy but necessary)
        }

        \li0 \he1-1NT-(\sp2)
        \ul {
            \li0 X - [18+] takeout
            \li0 2NT - [18-19] with stopper
            \li0 \cl3 - [18+]
        }
    }

\section{NT Interference} \label{3:4}

    Occasionally after 1-suit, our opponents will overcall at natural 1NT.  Our responses are as follows:

    \ul {
        \li0 \di1-(1NT)
        \ul {
            \li0 X - [9+] penalty.  All further X for penalty
            \li0 \cl2 - [6-9] 5+\cl{}, quality suit
            \li0 \di2 - [6-9] 4+\di (a fit)
            \li0 \he2 - [6-9] 6+\he{}s, quality suit
            \li0 \sp2 - [6-9] 6+\sp{}s, quality suit
        }
    }

    With distribution and values, good opponents will compete over our 1NT openings.  Over (X) and (\cl2), system is on, unless they show both majors.  Our XX is penalty oriented, X by responder shows values takeout-oriented though opener may X for penalties after.  If opponents show both majors, our defense is as follows:

    \ul {
        \li0 X - values, penalty interest
        \li0 \di2 - natural, signoff
        \li0 \he2 - [5-8] (4-5) or better in minors
        \li0 \sp2 - [8-9] (4-5) or better in minors
        \li0 2NT - lebensohl, forced puppet to \cl3.  Rebids:
        \ul {
            \li0 \di3 - signoff
            \li0 \he3 - stopper
            \li0 \sp3 - stopper
        }
        \li0 \cl3 - 5+\cl{}s, game forcing
        \li0 \di3 - 5+\di{}s, game forcing
    }

    When our opponents show a single suitor or two-suitor, 2-level rebids are natural and signoff, 2NT is lebensohl, and 3-level bids show a 5-card suit and game-forcing values.  An example after 1NT-(\he2) showing hearts and a minor:

    \ul {
        \li0 X - [8-9] 4\sp{}s
        \li0 \sp2 - 5+\sp{}s, signoff
        \li0 2NT - lebensohl.  5\sp{}s invite, weak minor, or 4\sp{}s.  After \cl3 puppet:
        \ul {
            \li0 \di3 - signoff
            \li0 \he3 - 4\sp{}s, GF with a stopper (slow shows)
            \li0 \sp3 - 5\sp{}s, invitational
        }

        \li0 \cl3 - 5+\cl{}s, GF
        \li0 \di3 - 5+\di{}s, GF
        \li0 \he3 - 4\sp{}s, GF without a stopper
        \li0 \sp3 - 5\sp{}s, GF
        \li0 3NT - to play
    }

    If 1NT is doubled for penalty (either as a 1NT overcall or an opening) the following escape is used:

    \ul {
        \li0 pass - forces opener to bid a 5-card suit or XX.  After XX: 
        \ul {
            \li0 pass - to play 1NTxx
            \li0 \cl2 - 4\cl{}s and higher suit
            \li0 \di2 - 4\di{}s and higher suit
            \li0 \he2 - 4\he{}s and a higher suit
        }
        \li0 XX - 5+ minor, transfer to \cl2
        \li0 \di2 - 5+\he{}s
        \li0 \he2 - 5+\sp{}s
    }
\section{Preempts} \label{3:5}

    Preempts are the some of most obstructive bids opponents can make; they remove valuable bidding space when we have a invitational or strong auction.  As such, we must sacrifice some slam accuracy in favor of reaching as many making games as possible.\n

    After a preempt, negative doubles are on until 4\he{}s, though 4-level negative doubles are passable (optional doubles).  Freebids at the 2-level are invitational and game-forcing at the 3-level, promising a 5-card suit. NT responses show a stopper, denying the majors.  Cues show support and game-forcing values. \n

    A cue of a 2-level preempt primarily asks for a stopper (Western), while a cue of a 3-level preempt shows support and slam aspirations.  Jump reponses are game-forcing and show a strong 6-card suit, denying support for the opener's suit.  Some examples:

    \ul {
        \li0 \di1-(\sp2)
        \ul {
            \li0 X - [9+] 4+\he{}s
            \li0 2NT - [11-12] with a stopper
            \li0 \cl3 - [12+] 5+\cl{}s, game-forcing
            \li0 \di3 - [7-10] 4+\di{}s
            \li0 \he3 - [12+] 5+\he{}s, game-forcing
            \li0 \sp3 - [12+] stopper ask or club support
            \li0 3NT - to play, stopper
        }
        \li0 \he1-(\sp3)
        \ul {
            \li0 X - [12+] minors, more of a stopper ask
            \li0 3NT - to play, stopper
            \li0 \cl4 - 5+\cl{}s, typically 6.
            \li0 \di4 - 5+\di{}s, typically 6.
            \li0 \he4 - to play
            \li0 \sp4 - Kickback (no cue available to set suit)
        }

        \li0 \cl1-(\he4)
        \ul {
            \li0 X - penalty
            \li0 \sp4 - 5+\sp{}s, to play
            \li0 4NT - slam invite in \cl{}
            \li0 \cl5 - to play
            \li0 \di5 - to play
        }
        
        \li0 \sp1-(\di3)
        \ul {
            \li0 X - [9+] 4\he{}s
            \li0 \he3 - [12+] 5+\he{}s
            \li0 \sp3 - [8-10] 3+\sp{}s
            \li0 3NT - to play, stopper
            \li0 \cl4 - 5+\cl{}s
            \li0 \di4 - slammish in \sp{}s
            \li0 \he4 - ***
            \li0 \sp4 - [11-15] 3+\sp{}s
        }
    }

    Sometimes, responder may not have the values or shape, and must pass.  At other times, they may have a 5 cards (or excellent 4 cards) in the preempted suit, interested in penalizing a preempt (and unable to bid 3NT).  Thus, opener should stretch to reopen with a X, around a good [13+].  As with the auction 1x-(1y)-p-(2y)-X, lebensohl is used.  An example:

    \ul {
        \li0 \di1-(\he2)-p-(p)-X
        \ul {
            \li0 pass - penalty
            \li0 \sp2 - [0-7] signoff
            \li0 2NT - [0-7], puppet to \cl3 for signoff
            \li0 \cl3 - [8-11] 5+\cl{}
            \li0 \di3 - [4-7] 4+\di{}
        }
    }
\section{Two-Suited Overcalls} \label{3:6}

    Whether in 2nd or 4th seat, our opponents can make 2-suited overcalls.  Since `our' suits and `their' suits are known, competition can become fierce with both sides potentially having a double fit so our methods must adapt. \n
    
    After a second seat two-suited overcall (Michaels, Unusual 2NT, another convention), pass shows [0-7], X shows [9+] with penalty interest, direct raises are competitive/preemptive, while cuebidding follows the `lower-lower' principle (a lower cuebid shows a lower suit). In order to bid 3NT, responder must have both suits stopped, so with only one, X and then cuebid the controlled suit. Some examples:

    \ul {
        \li0 \cl1-(\cl2) as both majors
        \ul {
            \li0 X - [9+] penalty suggestion, or forcing with 5+\di{}s
            \li0 \di2 - [6-9] 5+\di{}s (usually 6)
            \li0 \he2 - [10+] 4+\cl{}s (\he{} $<$ \sp{} and \cl{} $<$ \di{})
            \li0 \sp2 - [10+] 5+\di{}s, invitational
            \li0 \cl3 - [7-10] 4+\cl{}s, competitive
            \li0 \di3 - [0-9] 6+\di{}s, preemptive
            \li0 3NT - to play, 1.5+ stoppers.
            \li0 \cl{4/5} - [0-9] 5+\cl{}s, preemptive raise
        }
        \li0 \di1-(2NT) as \cl{}/\he{}
        \ul {
            \li0 X-(\cl3)-p-p
            \ul {
                \li0 X - penalty
                \li0 \he3 - \he{} stopper, no \cl{} stopper
            }
            \li0 X-(\he3)-p-p
            \ul {
                \li0 \sp3 - \cl{} stopper, no \he{} stopper
            }
            \li0 \cl3 - [10+] 4+\di{}s (\di{} $<$ \sp{})
            \li0 \di3 - [7-10] 4+\di{}, competitive
            \li0 \he3 - [10+] 5+\sp{}s, invitational
            \li0 \sp3 - [7-10] 6+\sp{}s, competitive
        }
        \li0 \sp1-(\sp2) as \he{}/minor
        \ul {
            \li0 X - penalty or [10+] no fit
            \li0 2NT - [11-12] \he{} stopper
            \li0 3m - [7-9] 5+ cards (typically 6)
            \li0 \he3 - [10+] 3+\sp{}s
            \li0 \sp3 - [7-9] 3+\sp{}s, competitive
            \li0 3NT - [12-15] \he{} stopper
            \li0 4z - [12+] 4\sp{}s, splinter
            \li0 \sp4 - [0-9] 4\sp{}s, preemptive
        }
    }

    When a suited overcall occurs in fourth seat, methods are similar.  Support X/XX are on (overcalls of \sp2 or higher are penalty-oriented), direct cues show a stopper and [17-19], 2NT shows [17-19] with both suits stopped (typically balanced).  Some examples:

    \ul {
        \li0 \cl1-\he1-(X)
        \ul {
            \li0 pass - no other bid, typically [12-14] balanced
            \li0 XX - \textit{exactly} 3\he{}s
            \li0 1NT - [12-14], partial stoppers in both suits
            \li0 \di2 - [17-19] diamond stopper
            \li0 \sp2 - [17-19] spade stopper
            \li0 2NT - [17-19] both stopped
            \li0 other - same as \cl1-\he1-?
        }
        \li0 \cl1-\sp1-(2NT)
        \ul {
            \li0 pass - [12-16] no other bid
            \li0 X - [14+] penalty-suggestion
            \li0 \cl3 - [15-17] 6+\cl{}s
            \li0 \di3 - [17+] stopper
            \li0 \he3 - [17+] stopper
            \li0 \sp3 - 4\sp{}s, [15-17]
            \li0 3NT - [17+] to play, both suits stopped
            \li0 \cl4 - [18+] 7+\cl{}s
            \li0 \di4 - [18+] 4\sp{}s, splinter
            \li0 \di4 - [18+] 4\he{}s, splinter
            \li0 \sp4 - [18-19] 4\sp{}s, balanced
        }
        \li0 \he1-1NT-(\he2)
        \ul {
            \li0 pass - no other bid, typically [12-14]
            \li0 X - [12-14] 6+\he{}s
            \li0 \sp2 - [18+] game force, typically no stopper
            \li0 2NT - [18+] spade stopper
            \li0 \cl3 - [17+] 4+\cl{}s
            \li0 \di3 - [17+] 4+\di{}s
            \li0 \he3 - [15-17] 6+\he{}s
        }
    }
\section{Slam Interference} \label{3:7}

    In slam auctions, the partnership cues first and second round controls (A, K, singleton, void).  Thus, opponents may make lead-directing doubles on this cue.  To show control type, redouble/pass is used:

    \ul {
        \li0 \sp1-2NT-\sp3-\cl4-(X)
        \ul {
            \li0 pass - denies A/void
            \ul {
                \li0 XX - 1st round
                \li0 4z - cue with 2nd round
            }
            \li0 XX - first round control
            \li0 3NT - non-serious, first round control
            \li0 4z - serious, first round control
        }
    }

    If opponents interfere with a keycard auction, 1430 step responses are used DOPI/ROPI style:

    \ul{
        \li0 \sp1-2NT-\sp3-4NT-\he5-
        \ul {
            \li0 X - 0 keycards (or 3)
            \li0 pass - 1 keycards (or 4)
            \li0 Step 1 - 2 without Q (or 5)
            \li0 Step 2 - 2 with Q (or 5)
        }
    }

    If X/XX or pass, the next step is queen ask.  If showing the queen, further bids are cues for grand.  If the kickback ask is available (usually after an X of kickback suit or with an X/pass response), it becomes the grand-slam invite/king-ask.
\chapter{Overcalling and Advancing} \label{4}
\section{Overview} \label{4:1}
    
    When the opponents open the bidding, it is important to both interfere with their strong auctions, as well as reach games (and rarely, slams) when we have the points or shape.\n

    Since most players play some variation of 2/1, these methods are dedicated against 2/1.  At the end of this chapter, a section is dedicated to artificial or short \cl1 and \di1 openings. \n

    The following conventions are used:
    \begin{itemize} \itemsep0em
        \item Takeout, power, penalty, responsive, optional X, ELC
        \item Michaels, unusual 2NT, leaping Michaels
        \item Cappelletti, Multi-Landy
        \item Gambling 3NT / cues
        \item lebensohl\n
    \end{itemize}

    Expected strength:
    \begin{itemize} \itemsep0em
        \li0 (1y)-1z - [(7+)-17]
        \li0 (1y)-1NT - [15-18]
        \li0 (2y)-2NT - [16-19]
        \li0 (1y)-p-(p)-1NT - [11-15] \n
    \end{itemize}

    Contents:
    \ul {
        \li0 \ref{4:1} - Overview
        \li0 \ref{4:2} - Suit-based Overcalls - 2nd Seat
        \li0 \ref{4:3} - Suit-based Overcalls - 4th Seat
        \li0 \ref{4:4} - NT Overcalls and Defense
        \li0 \ref{4:5} - Preempts
        \li0 \ref{4:6} - Two-Suited Overcalls
        \li0 \ref{4:7} - Balancing
        \li0 \ref{4:8} - Unnatural \cl1 and \di1 \newpage
    }

\section{Suit-based Overcalls - 2nd Seat} \label{4:2}

    After the opponents have shown suit-strength, it is important to interfere (thus overcalling light is possible) but without a good possibility of being doubled (not too light). \n

    Suit overcalls promise a 5-card suit (or better) with [(7+)-17] at the 1-level and [(10+)-17] at the 2-level.  X is the flexible forcing bid, showing either takeout for the majors or showing a powerhouse.  With [18+], first double, then bid a suit, cue, or NT.

    \ul {
        \li0 (\cl1)
        \ul {
            \li0 X - [11+], (33)+ majors (may have 2\di{}s) or [18+].  Usually 2-\cl{}s and (43)+ in majors
            \li0 1z - [(7+)-17] 5+ card suit with some quality
        }
        \li0 (\he1)
        \ul {
            \li0 X - [11+] for takeout, 3-4\sp{}s (typically 4).  Possible 5-4 in \di{}/\sp{}.
            \li0 \sp1 - [(7+)-17] 5+ cards, some quality
            \li0 2z - [(10+)-17] 5+ cards, good quality
        }
    }

    If 3rd seat interjects, X is responsive, showing values and either minors or majors.  Auctions like these are responsive: (\cl1)-X-(\he1)-X, (\he1)-X-(\he2)-X (showing the minors), or (\cl1)-X-(\cl3)-X showing the majors.

\subsection{Takeout Doubles}

    At the 1-level, X is not needed for penalty, and is instead for takeout, showing the majors, so even with a 5-card minor and 4-card major, one may choose to X instead to prioritize the other major. However, since X is not penalty, advancer is forced to respond with as few as [0] points.\n

    A NT advance shows values and a (typical) stopper, low suits show [0-7], single jumps show [8-11], and a cue shows [11+].  Double jumps below game are competitive/preemptive, and jumps to game show [12+] with a 5-card (or better) suit. \n
    
    \ul {
        \li0 (\cl1)-X-(p)
        \ul {
            \li0 pass - [10+] with strong clubs
            \li0 1z - [0-7] weak, 3+ suit
            \li0 1NT - [7-10] \cl{} typical stopper; no majors
            \li0 \cl2 - [12+] invitational+, information ask
            \li0 \di2 - [8-11] 4+\di{}s
            \li0 \he2 - [8-11] 4+\he{}s
            \li0 \sp2 - [8-11] 4+\sp{}s
            \li0 2NT - [11-12] with a stopper
        }

    When the auction returns to a strong overcaller, they may bid a new suit, cue, bid NT, or raise (a very weak hand) to show [18+] non-forcing.  After a jump advance, raises to game show a maximum takeout double [(14+)-17].

    One final adjunct: Equal Level Conversion (ELC).  After an opposing (1M) opening, X is either normal takeout, or ELC double.  An ELC double shows a 4-card major a 5+\di{}s.  If advancer bids clubs, a diamond bid \textit{at equal level} shows [11-17] 5+\di{}s, non-forcing. \n

        \li0 (\he1)-X-(p)
        \ul {
            \li0 \cl2 - [0-7] 4+\cl{}s (not a jump)
            \ul {
                \li0 \di2 - [11-17] ELC, 5+\di{}s
                \li0 \he2 - [18+] balanced w/o stopper, GF, or very strong club raise
                \li0 \sp2 - [18-21] 5+\sp{}s
                \li0 2NT - [19-21] balanced with stopper
                \li0 \cl3 - [15-17] 4+\cl{}s
                \li0 \di3 - [18-21] 5+\di{}s
            }
            \li0 \cl3 - [8-11] 4+\cl{}s 
            \ul {
                \li0 \di3 - [11-17] ELC 5+\di{}s
                \li0 \he3 - [18+] balanced w/o stopper or strong raise
                \li0 \sp3 - [18+] 5+\sp{}s
                \li0 \di4 - [18+] 5+\di{}s
            }
        }
    }

    When 3rd seat interjects, X shows 4-4 in the other suits (or minors after 1M-2M). Since there is no need to respond, weak hands are passed and the lowest-level suits show [8-11].  Jumps show a [6-9] 5-card suit with preemptive value.  Example:

    \ul {
        \li0 (\di1)-X-(\sp1)
        \ul {
            \li0 X - [8+] responsive, 4-4 in \cl{}/\sp{}
            \li0 1NT - [9-11] with stoppers, no fit
            \li0 \cl2 - [8-11] 4+\cl{}s
            \li0 \di2 - [12+] invitational+, information ask
            \li0 \he2 - [8-11] 4+\he{}s
            \li0 \sp2 - [8-11] 5+\sp{}s, natural since responder bid suit
            \li0 \cl3 - [6-9] 5+\cl{}s, preemptive
            \li0 \he3 - [6-9] 5+\he{}s, preemptive
        }
        \li0 (\he1)-X-(\he3)
        \ul {
            \li0 X - [10+] responsive 4-4 in \cl{}/\di{}
            \li0 \sp3 - [10-13] 4+\sp{}s
            \li0 3NT - to play
            \li0 \cl4 - [10-14] 5+\cl{}s
            \li0 \di4 - [10-14] 5+\di{}s
            \li0 \sp4 - to play
        }
    }

\subsection{Advancing Suits}

    After a wide ranging 1-level or 2-level overcall, game may still be on with a [8+] advancer.  New suits at the lowest level deny a fit (except possibly in (\cl1)-\di1-1M auctions), showing their own 5-card suit and [8-13] points.  A cue is invitational+ with a fit OR [14+] with another 5-card suit.  1/2/3NT bids show [9-11], [12-14], [15-18] almost certainly with a stopper. \n

    While typically showing invitational+ and a fit, a cue is simply a forcing bid.  Single jumps are fit-showing and invitational, double jumps are splinters.  However, since cues show stronger raises, jump raise are freed up to be preemptive raises.  An example:

    \ul {
        \li0 (\cl1)-\di1
        \ul {
            \li0 \he1 - [8-13] no fit, 5+\he{}s
            \li0 \sp1 - [8-13] no fit, 5+\sp{}s
            \li0 1NT - [9-11] no fit
            \li0 \cl2 - [10+] 3+\di{}s or [14+]
            \ul {
                \li0 \di2 - [8-10] any minimum
                \li0 2M - [11-17] 4-card suit (help suit if major overcalled)
                \li0 2NT - [11-14] with a stopper
                \li0 \cl3 - [14+] cue, GF
                \li0 \di3 - [11-14] 6+\di{}s
                \li0 3NT - [15-17] with a stopper
            }
            \li0 \di2 - [8-10] 3+\di{}
            \li0 2M - [10-13] 5+ suit, 4\di{}s
            \li0 2NT - [12-14] no fit, usual stopper
            \li0 \cl3 - [6-9] 4+\di{}s, mixed raise
            \li0 \di3 - [4-8] 4+\di{}s, unbalanced
            \li0 3M - [14+] 4+\di{}s, splinter
            \li0 3NT - [15-18] no fit, stopper
        }
    }

\section{Suit-based Overcalls - 4th Seat} \label{4:3}

    4th seat overcalling is nearly identical to 2nd seat overcalling.  However, since partner has denied strength (by not overcalling), overcalls promise better suits and more strength.  [9-17] for 1-level overcalls, and [12-17] for 2-level, with good suit quality. \n

    Since opponents have bid two suits (or 1 in examples such as \he1-\he2 or \sp1-1NT), X is still takeout, but for the other suits.  After a NT response or raise, X is identical to 2nd seat, but with [13+]. \n

    After a strong action (2/1, GF raise, etc), doubles are takeout for natural suits, lead-directing for artificial bids, and new suits are strong and long. \n
    
    It is also important to note that responder only promises a 4-card suit, and opener minors only promise a 3-card suit.  Thus, cues are \textit{natural} showing 6+ cards and great suit quality.  Some examples:

    \ul {
        \li0 (\cl1)-p-(\sp1)
        \ul {
            \li0 X - 4-4 or better in \di{}/\he{}
            \li0 2-suit - [12-17] 5+ cards, great quality
        }
        \li0 (\he1)-p-(1NT)
        \ul {
            \li0 X - [13+] takeout of \he{}s, 4\sp{}s
            \li0 2z - [12-17] 5+ cards, great quality
        }
        \li0 (\sp1)-(\di2)
        \ul {
            \li0 X - [13+] typically 4-5
            \li0 \he2 - [12+] 6+\he{}s
            \li0 \cl3 - [12+] 6-7+\cl{}s
        }
    }

    After a 4th seat X, 2nd seat's bid of the opponents (quasi) suits are \textit{natural} and to play, showing no support for either of overcaller's suits.  X is responsive, but still limited by overcaller's pass.  Occasionally overcaller will be stronger than expected [12-14], balanced with openers suit. \n

    The auction (1M)-p-(2M)-X is special, since X can be highly flexible.  The opponents have a known major fit, and 4th seat is short.  Thus, it is more important to find a good partial, rather than game.  Games can be bid directly by the massive overcaller or a good advancer.  For partial-hunting, 2NT is employed as `scrambling', asking the partnership to bid up 4-card suits to find a fit, naturally implying direct bids are show a 5+ card suit.

    \ul {
        \li0 (\sp1)-p-(\sp2)-X
        \ul {
            \li0 2NT - scrambling, bid 4-card suits up the line.  4-card suits
            \li0 \cl3 - 5+\cl{}s
            \li0 \di3 - 5+\di{}s
            \li0 \he3 - 4+\he{}s (typically 5)
        }
    }
\section{NT Overcalls and Defense} \label{4:4}

    Sometimes the opponents will open when one holds a strong balanced hand.  1NT shows this hand \textit{and a stopper}.  Without a stopper or stronger, double and show the hand later.  After this, system is on.  Note that overcaller can have up to 18 points, so point ranges slightly decrease.\n

    However, when the opponents open a NT, it is important to interfere or reach game ourselves, depending on their point range.  For the purposes of definition, a strong NT is any NT containing [16].  [15-17], [14-16] are considered strong, while [13-15] and lower are considered weak. \n

    Opposite a strong NT, a modified version of Multi-Landy is used:
    \ul {
        \li0 X - [10+] 4-5 in a minor and major, or 6+\di{}s
        \ul {
            \li0 pass - [12+] penalty
            \li0 \cl2 - minor tolerance
            \li0 \di2 - major tolerance
            \li0 2M - 6+ suit, to play
        }
        \li0 \cl2 - [9+] both majors
        \ul {
            \li0 \di2 - asking for longer major, or spade invite
            \li0 2M - preference
        }
        \li0 \di2 - [9+] 6+ cards in an unspecified major
        \ul {
            \li0 \he2 - pass or correct
            \li0 \sp2 - to play
        }
        \li0 \he2 - 5+\he{}s and 4+ unspecified minor
        \ul {
            \li0 2NT - show minor
        }
        \li0 \sp2 - 5+\sp{}s and 4+ unspecified major
        \li0 2NT - 5-5 in minors
    }

    Against a weak NT, X is needed as penalty, so the Cappelletti defense is used:
    \ul {
        \li0 X - [16+] penalty
        \li0 \cl2 - [9+] a single suitor
        \li0 \di2 - [9+] both majors
        \li0 \he2 - hearts and a minor
        \li0 \sp2 - spades and a minor
        \li0 2NT - 5-5 in minors
    }

    When vulnerable, the overcaller should have more shape or points.  When favorable, overcalls may have 4-4 shape, though this should be uncommon and reserved for [12+] or 4441 hands. \n

    In the balancing seat of a \textbf{strong} NT: (1NT)-p-(p), only Landy is used (all other naturals), with X for clubs.  This allows 2nd seat to pass X for penalty when holding a strong hand:
    \ul {
        \li0 X - [9+] 5+\cl{}s
        \li0 \cl2 - [9+] majors
        \li0 2z - [9+] 5+ suit, natural
    }

\section{Preempts} \label{4:5}

    As seen during \ref{3:5}, preempts can be powerful ways to disrupt the auction.

\subsection{Jump Overcalls}
    
    Since [18+] overcalls are through power doubles, jumps are freed up to be preemptive.  If partner is passed, jumps may be weaker than usual, though opposite an unpassed partner, one should bid according to the rules of opening a preempt. \n

    A 3NT jump overcall is gambling, showing AKQxxxx or better and a stopper in the overcalled suit.  Responses are identical to opening 3NT. \n

    In case of a strong responder, advance with system on assuming overcaller opened.  2NT is still OGUST, and strong raises can be a cue. RONF is also on, meaning a raise of a preempt is \textit{not} invitational; raises are meant to further the preemptive.

\subsection{Overcalling Preempts}

    When the opponents deal and open a preempt, it is critical that games can be made.  However, no shape or strength information about our side is known.  As such, we must sacrifice some slam-searching tools and raises in favor of making games. \n

    NT overcalls are natural, showing strength and a stopper.  Direct overcalls have opening strength with a 5-card (or longer) suit, while jump overcalls are strong single suitors (GF) or artificial (discussed later).  Cues ask for a stopper, showing a long running minor, while a direct bid of 3NT shows a long minor with a stopper. X is takeout up through \di4.  In order to deal with difficult hands, lebensohl (slow-shows) and leaping Michaels are used:

    \ul {
        \li0 (\di2)
        \ul {
            \li0 X - takeout [18+]
            \ul {
                \li0 2M - signoff
                \li0 2NT - lebensohl, relay to \cl3
                \ul {
                    \li0 \cl3 - accept transfer
                        \li0 \di3 - GF ask, stopper
                        \li0 \he3 - [8-11] 5+\he{}s, stopper
                        \li0 \sp3 - [8-11] 5+\sp{}s, stopper
                    
                    \li0 \di3 - massive hand, no stopper
                    \li0 \he3 - [19+] 5+\he{}s, GF
                    \li0 \sp3 - [19+] 5+\sp{}s, GF
                    \li0 3NT - to play
                }
                \li0 \cl3 - 4+\cl{}s, GF
                \li0 \di3 - GF ask, no stopper
                \li0 \he3 - [8-11] 5+\he{}s, no stopper
                \li0 \sp3 - [8-11] 5+\sp{}s, no stopper
            }
            \li0 \he2 - [11-17] 5+\he{}s
            \li0 \sp2 - [11-17] 5+\sp{}s
            \li0 2NT - [16-19] with a stopper
            \li0 \cl3 - [12-17] 5+\cl{}s
            \li0 \di3 - long minor, stopper ask
            \li0 \he3 - [18+] 6+\he{}s strong single suitor, GF
            \li0 \sp3 - [18+] 6+\sp{}s, strong single suitor, GF
            \li0 3NT - long minor, to play
            \li0 \cl4 - 5+\cl{}s and 5+ unspecified major, GF
            \li0 \di4 - (5-5)+ in majors
            \li0 4M - [13-17] 8+ suit (with stronger, bid 3M forcing)
        }
        \li0 (\sp2)
        \ul {
            \li0 X - takeout (4\he{}s) or [18+]
            \ul {
                \li0 2NT - lebensohl.  After \cl3 relay:
                    \li0 pass - [0-7] to play
                    \li0 \di3 - [0-7] to play
                    \li0 \he3 - [0-7] to play
                    \li0 \sp3 - [12+] 4\he{}s, stopper
                
                \li0 \cl3 - [8-11] 4+\cl{}s
                \li0 \di3 - [8-11] 4+\di{}s
                \li0 \he3 - [8-11] 4+\he{}s
                \li0 \sp3 - [12+] 4+\he{}s, no stopper
            }
            \li0 3z - [12-17] 5+ suit
            \li0 \sp3 - long minor, stopper ask
            \li0 \cl4 - 5+\cl{}s and 5+\he{}s, GF
            \li0 \di4 - 5+\di{}s and 5+\sp{}s, GF
        }
        \li0 (\cl3)
        \ul {
            \li0 X - takeout
            \ul {
                \li0 \cl4 - major ask or very strong
            }
            \li0 \cl4 - both majors
            \li0 \di4 - 5+\di{}s and 5+ unspecified major, GF
        }
        \li0 (\he3)
        \ul {
            \li0 \cl4 - natural (not a jump)
            \li0 \di4 - natural (not a jump)
            \li0 \he4 - 5+\sp{}s, 5+ unspecified minor, GF
        }
        \li0 (\sp4)
        \ul {
            \li0 X - [13+] optional/penalty
            \li0 4NT - strong two suitor
            \li0 5z - natural
        }
    }


\section{Two-Suited Overcalls} \label{4:6}

    When holding a shapely hand, auctions may rise too fast to show both suits naturally.  Additionally, when the opponents have game, showing shapely hands can help our partner lead or make sacrifices.  We employ Michaels cuebid to show 5-5s with a major, and Unusual NT to show a 5-5 in the lowest bid suits. \n

    However, since these bids are always at the 2-level or higher, vulnerability must be considered (the partnership is forced to the 3-level on a misfit).  When nonvulnerable, these bids show [6-11] or [16+].  When vulnerable, [12-17] or [22+] (a 6-point differential).  A memory aid: vulnerable requires opening strength [12+], non-vul is just under opening strength [11-]. \n
    
    These point ranges are not exact.  Additional length and no wastage drastically improves the hand.  For instance, having \spades{KJT9xx} \hearts{AJT9xxx} \di{} \cl{x} after a \di1 opening has massive playing strength with only 5 losers.\n
    
    If at the 2-level, 2NT is a minor asking bid (typically weak).  Some example bidding sequences:

    \ul {
        \li0 (\cl1)
        \ul {
            \li0 \cl2 - (5-5)+ in \he{}/\sp{}
            \li0 2NT - 5\he{}, 5+\di{}s
        }
        \li0 (\di1)-p-(\di2) (strong)
        \ul {
            \li0 \di3 - (5-5)+ in \he{}/\sp{}
            \li0 2NT - 5\he{}, 5+\cl{}s
        }
        \li0 (\he1)
        \ul {
            \li0 \sp2 - exactly 5\sp{}s, 5+ unspecified minor.
            \li0 2NT - (5-5)+ in \cl{}/\di{}
        }
        \li0 (\sp1)-p-(\sp3) (weak)
        \ul {
            \li0 \sp2 - exactly 5\he{}s, 5+ unspecified minor.
            \li0 3NT - \textbf{natural, to play}
            \li0 \sp4 - (5-5)+ in hearts and an unspecified minor.
            \li0 4NT - (5-5)+ in \cl{}/\di{}
        }
    }

    Note that when showing one major and a minor (at a sufficiently low level) through 2NT or Michaels, one holds \textit{exactly} 5 cards in the major.  With a 6-card major and a 5-card minor, bid the major first.  When showing both majors, this restriction is unnecessary, as responder can choose either major game. \n

    Unusual NT may occur at the 1 or 3-level by a passed hand.  At the 1-level (Sandwich NT), this only shows 5-4 shape, and fewer points (and defensive value) than X.  A sample auction: p-(\cl1)-p-(\sp1)-1NT.

\section{Balancing} \label{4:7}

    The balancing seat (and bid) is the bid that ends the auction with a pass (excluding 4 passes).  Since letting the opponents buy the contract at the 1-level is usually a poor choice, especially when ones partner has a stack of the opener's suit.  Thus, overcalling requirements are lower, including for 1NT (and 2NT over preempts). \n

    As a general rule, add a (worthwhile) 3 HCP when balancing, and bid accordingly.  Thus, responding to a balanced overcall requires more strength or fit, but is identical to responding to a simple overcall.  Doubles are even more flexible. \n
    
    Since responder is passed, a jump overcall is not preemptive.  Thus, jumps show a good suit with 2-level overcall strength. \n

    Balancing over a strong NT has already been discussed (Landy and X=\cl{}).  Excluding that auction:

    \ul {
        \li0 (\cl1)-p-(p)
        \ul {
            \li0 pass - some club length (opener may have [18-19] balanced with 3\cl{})
            \li0 X - [8+] club shortness or [16+]
            \li0 1z - [7-15] 5+ suit (possibly an excellent 4)
            \li0 1NT - [11-15] balaced.  \textbf{System on}
            \li0 \cl2 - Michaels.  Do \textbf{not} include the extra 3 HCP
            \li0 2z - [12-17] 6+ suit
            \li0 2NT - Unusual, 5-5 in \di{}/\he{}
        }
        \li0 (\he1)-p-(p)
        \ul {
            \li0 X - [7+] 2-\he{} (weaker since opener has a suit and escape in \sp1)
        }
    }

    One can also balance over preempts, with the same 3-point differential. Thus, advancer must be careful to not overbid, as overcaller has `borrowed' a king. \n
    
    Doubles still serve as power doubles, showing [16+], Unfortunately, with a [15+] NT, one must make a balancing double and rebid NT.

\bidsection{Unnatural \cl1 and \di1}{Unnatural 1♣ and 1♢}\label{4:8}

    When our opponents play a non-2/1 or standard system, it typically revolves around a strong \cl1 or \di1 with a nebulous (quasi) \di1 or \cl1 respectively.  A nebulous minor is one that can be opened with 2 or less cards in the suit (excluding 4432 exactly).  For purpose of this section, we assume \cl1 is [16+] and \di1 is nebulous. \n

    Against a strong [16+] 1m opening, the Mathe convention is used:

    \ul {
        \li0 X - both majors
        \li0 1NT - both minors
    }

    Additionally, most strong club and diamond players expect interference, thus 2-level suits or weak jump overcalls require some suit quality or a higher suit for escape if penalized.  Cuebids over strong, artificial bids are \textit{natural}, allowing for club hands to interfere.\n

    Nebulous bids, however, are far more problematic.  Since either partner may have diamonds (especially opposite when opener promises as few as 0), bids must be repurposed.  With \he2 being the weakest jump preempt, it is repurposed as Michaels, and \di3 as a strong Michaels.  In lieu of (\di1)-\di2-(p)-\cl3 is a cue-raise (as opener likely will certainly have clubs).
    
    \ul {
        \li0 (\di1)-short as (2-) or less
        \ul {
            \li0 \di2 - 5+ natural (further responses now function as if \cl1 was opened instead)
            \ul {
                \li0 \cl3 - cue raise
            }
            \li0 \he2 - [6-11] Michaels (non-vulnerable range used)
            \li0 2NT - Unusual, \textit{exactly} 5\he{}s and ambiguous 5-card minor
            \ul {
                \li0 \cl3 - pass-or-correct
            }
            \li0 \di3 - [16+] Michaels
        }
    }

    Doubling a nebulous minor will promise the majors, and not the `unbid' minor.

\chapter{Carding} \label{5}

\section{Leads}

    Against NT contracts, leads are fourth-best from the longest suit.  Suit contracts instead use 3rd/5th leads. The common honor leads are as follows (multiple underlines show varied leads):

    \newcommand{\uu}[1]{\underline{#1}}
    \begin{center}
        \begin{tabular}{ |c|c| } 
            \hline
            Suit & NT \n
            \hline
            \uu{x}x & \uu{x}x\n
            xx\uu{x} & \uu{x}xx ***\n
            xx\uu{x}x & xxx\uu{x}\n
            xxxx\uu{x} & xxx\uu{x}x\n
            Hx\uu{x} & Hx\uu{x}\n
            Hx\uu{x}x & Hxx\uu{x}\n
            Hxxx\uu{x} & Hxx\uu{x}x\n
            \uu{A}Kx & \uu{A}Kx\uu{x}\n
            \uu{K}Qx & \uu{K}QJx\n
            \uu{Q}Jx & K\uu{Q}T9\n
            \uu{J}Tx & \uu{Q}JTx\n
            \uu{T}9x & \uu{J}T9x\n
            K\uu{J}Tx & A\uu{Q}Jx\n
            K\uu{T}9x & A\uu{J}Tx\n
            Q\uu{T}9x & K\uu{T}9x\n
            A\uu{K} & Q\uu{T}9x\n
            \hline
        \end{tabular}
    \end{center}

\section{Signals}

    If partner leads a suit, one shows upside-down attitude (2=strong, 10=weak).  If declarer leads a suit, show upside-down count (2=even, 10=odd).  If declarer leads a trump, show suit preference (if useful). \n

    For the first discard, odd-even discards are used.  An odd numbered discard shows strength in the suit, while an even discard discourages and shows suit preference, a low even shows preference for the lower-ranking non-trump suit, a higher discard shows preference for a higher-ranking non-trump discard.  In case of NT, show high/low relative to suit played.\n

    Honor discards show complete control of a suit, but when a suit control is on the board (singleton, Kx, Qxx after AK, etc.), show suit preference.\n

\end{document}