% todo - ctrl-f for ***

\documentclass[12pt]{report}
\large
\hbadness=99999

\usepackage{extsizes}
\usepackage{amssymb,amsfonts,amsmath}
\usepackage{mathrsfs}
\usepackage[unicode=true]{hyperref}
\usepackage[capitalise]{cleveref}
\usepackage{grbbridge}
\usepackage[utf8]{inputenc}
\usepackage{geometry}
\geometry{a4paper, total={170mm,257mm}, left=20mm, top=20mm,}

\newcommand{\n}{\\}
\newcommand{\ol}[1]{\begin{enumerate}#1\end{enumerate}}
\newcommand{\ul}[1]{\begin{itemize}#1\end{itemize}}
\newcommand{\li}{\item[~]}
\newcommand{\bidsection}[2]{\section{\texorpdfstring{#1}{#2}}}


\title{\bf{2/1 Game Forcing}}
\author{Daniel Schuck, Han-Mo Ou}

%%%%%%%%%%%%%%%%%%%%%%%%%%%%%%%%%%%%%%%%%%%%%%%%%%%%%%%%%%%%%%%%%%%%%
%%%%%%%%%%%%%%%%%%%%%%%%%%%%%%%%%%%%%%%%%%%%%%%%%%%%%%%%%%%%%%%%%%%%%
%%%%%%%%%%%%%%%%%%%%%%%%%%%%%%%%%%%%%%%%%%%%%%%%%%%%%%%%%%%%%%%%%%%%%

\begin{document}

\maketitle
\newpage

\chapter{Introduction}
\section{Biography}

    Daniel Schuck, a current junior at University of Illinois Urbana-Champaign.  An avid overbidder and the primary author of this book.  System and convention design is fascinating to me, despite cardplay being far more important.
\n\n

\section{Motivation}

    This book exists to formalize our 2/1 bidding system to both avoid misunderstanding and optimize game and slam searching. ***
\n\n

\section{Structure}

    This system is built off a fairly standard 2/1 game-forcing system.  It features artificial fit-showing responses after opener (or responder) show a 4-card minor after transfers, Gazzilli, etc.\n

    Two common motifs are used for artificial raises:
    
    The first motif: if one must decide between showing a (potential) major fit and a minor fit, show the major first. For instance, after 1NT-\di2-\he2-\cl3, a club fit is confirmed via \di3, which necessarily \textit{denies} a heart fit, as opener must show the major fit first.\n

    The second motif is surrogacy.  In case where opener needs to show some suit feature (shortness, specific K, etc.) but that suit is the asking bid (ie, \di5 for \cl{} Kickback) or would commit to game/slam, the next lowest NT bid is used as the surrogate bid (ie, 5NT to show the K of diamonds).

\newpage

\chapter{Openings and Continuations}
\section{Overview} \label{2:1}
    HCP notation: [$x$-$y$] indicates HCP range from `$x$' to `$y$' inclusive, with ($x$+) and ($y$-) indicating an upgraded or downgraded hand respectively.  [(14+)-17] is a good 14 to any 17 count. Table of openings is as follows:
    
\begin{center}
    \begin{tabular}{ |c|c|c|c| } 
        \hline
        \cl1                     & [11-21]    & 3+\cl{}s             & open with 3-3 in minors \n
        \di1                     & [11-21]    & 3+\di{}s             & 3-cards only when 4432  \n
        \he1                     & [11-21]    & 5+\he{}s             & could have 6-card minor \n
        \sp1                     & [11-21]    & 5+\sp{}s             & could have 6\he{}s      \n
        1NT                      & [(14+)-17] & (semi)balanced       & could have 5-card major \n
        \cl2                     & [22+]      & or 8.5 tricks        & game-forcing            \n
        2\di{}/\he{}/\sp{}       & [4-10]     & 6-card suit          & preemptive/constructive \n
        2NT                      & [20-21]    & (semi)balanced       & not 9-cards in major    \n
        3\cl{}/\di{}/\he{}/\sp{} & [4-10]     & 7-card suit          & highly preemptive       \n
        3NT                      & [9-11]     & AKQxxxx in minor     & no outside stoppers     \n
        4\cl{}/\di{}             & [7-14]     & 8+ major, 8+ tricks  & transfer up-2           \n
        4\he{}/\sp{}             & [4-10]     & 8+ major, weak suit  & usually fairly weak     \n
        \hline
    \end{tabular}
\end{center}
Contents:
\ul {
    \li \ref{2:1} - Overview
    \li \ref{2:2} - \cl1
    \li \ref{2:3} - \di1
    \li \ref{2:4} - \he1
    \li \ref{2:5} - \sp1
    \li \ref{2:6} - 1NT - [(14+)-17]
    \li \ref{2:7} - \cl2 - Game-Forcing
    \li \ref{2:8} - 2\di{}/\he{}/\sp{}
    \li \ref{2:9} - 2NT - [20-21]
    \li \ref{2:10} - 3\cl{}/\di{}/\sp{}
    \li \ref{2:11} - 3NT - Gambling
    \li \ref{2:12} - 4\cl{}/\di{}/\he{}\sp{} - NAMYATS
}
\newpage
\bidsection{\cl{1}}{1♣} \label{2:2}

    When a major is un-openable, \cl1 is opened with either longer clubs, or 3-3 in the minors. this necessarily puts strain on the minors: \cl1 will have only 3-cards 16\% of the time.  While low, it is not insignificant.\n
    
    Another factor to consider with 1m openings is the 1NT strength.  Since this system uses a [14(+)-17] NT, the weak [12-14] NT hands naturally fall into the 1m openings.  So if a 1m is balanced, it will have either [12-14] or [18-19] HCP.\n

    The most important style used over \cl1 is Walsh, a bypass of diamonds (potentially a 6-card suit) to show a 4-card major suit.  If responder is game-forcing, they shall not bypass diamonds, as they can reverse into the major later.\n

    Conventions (or styles) used include Walsh, inverted minors, splinters, artificial reverses, blackout after reverses (bid cheaper of 4th suit or 2NT), splinters, weak jump shifts, and xyz (excluding \cl1-\di1-\he1 auction, bids are natural with \sp1 as GF). \n
    
    The full structure is as follows:

    \ul {
        \li \di1 - [6+] Walsh style, no major unless GF
        \ul {
            \li \he1 - [11-18] 4+\cl{}s, 4+\he{}s, unbalanced
            \li \sp1 - [11-18] 4+\cl{}s, 4+\sp{}s, unbalanced (denies hearts)
            \li 1NT - [12-14] balanced or semi-balanced.  May have 4-card major
            \li \cl2 - [11-14] 6+\cl{}s
            \li \di2 - [11-14] 4+\di{}s (necessarily 5+\cl{}s)
            \li \he2 - [19+] Game forcing.  Either a club single suitor or 4+\he{}s
            \li \sp2 - [19+] 4+\cl{}s, 4+\sp{}s.  Game forcing
            \li 2NT - [18-19] balanced or semi-balanced.  May have support/majors
            \li \cl3 - [15-17] club single suitor.  Typically unbalanced
            \li \di3 - [15-17] 4+\di{}s (necessarily 5+\cl{}s)
            \li \he3 - [18+] 4+\di{}s, splinter
            \li \sp3 - [18+] 4+\di{}s, splinter
        }

        \li \he1 - [6+] 4+\he{}s, possible \di{} canapé if weak
        \ul {
            \li 1NT - [12-14] balanced or semi-balanced.  Use xyz (shown in \di1 structure)
            \li \sp1 - [11-18] 4+\sp{}s.  Can be balanced.  Use xyz
            \li \cl2 - [11-14] 5+\cl{}s, almost always 6
            \ul {
                \li \di2 - artificial GF.  Either 5+\he{}s or slammish in clubs.
                \li \he2 - signoff
                \li \sp2 - natural responder reverse, GF
                \li 2NT - [11-12] invite
                \li \cl3 - [10-12] invite with support
                \li \he3 - [10-12] 6+\he{}s, invite
            }
            \li \di2 - [18+] artificial.  Natural \di{} reverse or single suitor
            \ul {
                \li \he2 - 5+\he{}s
                \li \sp2 - blackout, very weak hand (all other bids GF)
                \ul {
                    \li 2NT - [18-19] diamond reverse (surrogate)
                    \li \cl3 - [18-19] single suitor
                    \li \di3 - [20-21] 4-6 or better reverse.  Looking for 5m
                    \li 3NT - to play
                }
                \li 2NT - balanced GF
                \li \cl3 - 4+\cl{}s
                \li \he3 - 6+\he{}s, slammish
            }
            \li \he2 - [11-14] 3+\he{}s, almost never 3-card support.
            \ul {
                \li Suit - game try, promises 5\he{}s
                \li 2NT - non-forcing invite
                \ul {
                    \li pass - 3\he{}s, deny
                    \li \he3 - 4\he{}s, deny
                    \li 3NT - 3\he{}s, accept
                    \li \he4 - 4\he{}s, accept
                }
                \li 3NT - pass-or-correct
            }
            \li \sp2 - [19+] 4+\sp{}s, game forcing
            \li 2NT - [18-19] balanced or semi-balanced
            \ul {
                \li \cl3 - transfer to \di{3} for signoff or opener's minor slam try.  After \di3 (passable):
                \ul {
                    \li \he3 - 4-5 in majors, pass or correct to \sp3
                    \li \sp3 - 4-6 in majors, pass or correct to \he4
                    \li 3NT - slam try in \cl{}s
                }
                \li \di3 - checkback
                \li \he3 - 6-card suit, slammish
                \li \sp3 - 4-5 in majors
                \li Games - to play
            }
            \li \cl3 - [15-17] 6+\cl{}s
            \li \di3 - [15-17] 4+\he{}s, mini-splinter
            \li \he3 - [15-17] 4+\he{}s, unbalanced (by inference: spade shortness)
            \li \sp3 - [18+] game-forcing splinter
        }
        \li \sp1 - [6+] 4+\sp{}s, denies 4\he{}s unless longer spades
        \ul {
            \li \di2 - [18+] artificial.  Either natural diamond reverse or single suitor
            \ul {
                \li \he2 - blackout
                \ul {
                    \li \sp2 - 3-card support, non-forcing
                }
            }
            \li \he2 - [18+] 4+\he{}s 5+\cl{}s, reverse
            \ul {
                \li 2NT - blackout, relay to \cl3 unless GF
            }
            \li \di3 - [15-17] mini-splinter
            \li \he3 - [15-17] mini-splinter
        }
        \li 1NT - [7-10] no major, balanced
        \ul {
            \li \cl2 - signoff
            \li \di2 - artificial.  Natural reverse or single suited
            \li \he2 - natural reverse
            \li \sp2 - natural reverse
            \li 2NT - [15-16] invitational
            \li 3NT - [17+] signoff
            

        }
        \li \cl2 - [10+] 4+\cl{}s, strong raise or [16+] balanced without major
        \ul {
            \li \di2 - [12-14] natural
            \li \he2 - stopper or awkward GF
            \li \sp2 - stopper
            \li 2NT - [12-14] balanced
            \li \cl3 - [12-14] 5+ \cl{}s
            \li \di3 - [15+] 5+ \cl{}s, splinter
            \li \he3 - [15+] 5+ \cl{}s, splinter
            \li \sp3 - [15+] 5+ \cl{}s, splinter
            \li 3NT - [18-19] balanced
        }

        \li \di2 - [0-5] very weak jump
        \li \he2 - [0-5] very weak jump
        \li \sp2 - [0-5] very weak jump
        \li 2NT - [11-12] invite
        \li \cl3 - [6-9] 5+\cl{}, preemptive.  Correctable with [18-19]
        \li \di3 - 5+\cl{}s, splinter (deny majors)
        \li \he3 - 5+\cl{}s, splinter (deny major)
        \li \sp3 - 5+\cl{}s, splinter (deny majors)
        \li 3NT - [13-15] balanced
    }

    Passed hands respond identically, but cannot make forcing bids (ie \cl2 is passable).  Since responder cannot have a game force, xyz \di2 shows a maximum invite, generally an [11(+)-12] count that is nearly openable, while \cl2 (still transfer) shows [10-11].
\newpage

\bidsection{\di{1}}{1♢} \label{2:3}

    \di1 is the simpler of the minor openings.  It promises 3+\di{}s, but only has 3 when holding exactly 4432 shape.  Thus, opener will hold 4+\di{}s 96\% of the time.  Additionally, there is no suit to bypass, so no Walsh treatment is necessary.  As most of the \cl1 structure remains in the \di1 structure, only the differences will be shown:

    \ul {
        \li \he1 - 4+\he{}s
        \ul {
            \li \sp1 - [11-18] 4+\sp{}s
            \ul {
                \li 1NT - [7-10] minimum
                \li \cl2 - relay to \di2 for any invite or signoff.  Rebids:
                \ul {
                    \li \he2 - 5\he{}s
                    \li \sp2 - 4\sp{}s and 5\he{}s
                    \li 2NT - [11-12] natural
                    \li \cl3 - [10-12] 5+\clubs{} (typically 6)
                    \li \di3 - [10-12] invite with fit
                    \li \he3 - [10-12] 6\he{}s
                }

                \li \di2 - artificial GF
                \li \he2 - [6-10(-)] 5+\he{} (typically 6)
                \li \sp2 - natural reverse
                \li 2NT - transfer to \cl3 (6-card suit)
                \li 3(their suit) - slam try in suit
                \li 3(other suit) - 6+ cards, slam try.  3NT denies fit
            }
            \li 1N - [12-14] balanced
            \ul {
                \li \cl2 - relay to \di2 for invite or signoff
                \li \di2 - artificial GF
                \li \he2 - signoff
                \li \sp2 - natural reverse
                \li 2NT - transfer to \cl3 (6-card suit)
                \li 3z - flag suit, slam try
            }
            \li \sp2 - natural jump shift in \sp{}, or single suitor
            \ul {
                \li 2NT - blackout
                \li \cl3{} - waiting bid (typically 5\he{}s)
                \li \di3{} - 3+\di{}s, support
                \li \he3{} - 6+\he{}s
            }
            \li \cl3 - [19+] 4+\cl{}s, GF
        }
        \li \sp1 - 4+\sp{}s, denies 4+\he{}s unless longer spades
        \li 1NT - [7-10] no major, may have a 6\cl{}s
        \li \cl2 - 2/1, 5+\cl{}s.  Does not deny major.
        \li \cl3 - [10-12] 6+\cl{}s, good suit quality.  Denies majors
    }
\newpage

\bidsection{\he{1}}{1♡} \label{2:4}

    Major openings require a 5+ card suit in every seat and vulnerability, requiring [11-21] HCP.  In 3rd seat, one may open with [10] HCP under protection of \cl2 Drury.  \n
    
    Since the higher suit of equal length is opened, \he{1} denies 5\sp{}s, unless opener shows a 5-6 (or better) in the majors. However, with 5-6 in a major-minor, the major is \textbf{always} opened.  Gazzilli allows showing both [14(+)-16] and [17+] of this shape.  This naturally relieves strain on minor-major reverses, which promise exactly a 4-card major (unless artificial).\n

    The most important motif for slam bidding (in any opener) is that as soon as a major fit is discovered, the major is flagged and the partnership is committed to playing in the major, unless both players are balanced (ie 3334 opposite 5332).  Serious cues, non-serious 3NT, Kickback, and EKC are on after a suit is flagged.  \n

    Conventions over 1M include a semi-forcing 1NT, \textit{Gazzilli}, Jacoby 2NT, Reverse Bergen, balanced 3NT raise, non-serious 3NT, splinters, Kickback (1430), and EKC (0314). Note that jumps to 4M are \textit{always} preemptive as responder has several ways to show support and cuebid/keycard.  A jump in the opposite major shows a long (typically self sufficient) suit with [4-8] HCP.  \n

    The structure is as follows:

    \ul {
        \li \sp1 - 4+\sp{}s, denies \he{} support unless GF.  Forcing
        \ul {
            \li 1NT - [12-14] minimum, balanced or semi-balanced.  Denies 4\sp{}s
            \li \cl2 - [11-16] with 4+\cl{}s or any [17+]
            \ul {
                \li \di2 - [8+] any, GF relay opposite strong hand
                \ul {
                    \li \he2 - [11-16] with 4+\cl{}s.  Use minor raise structure
                    \li \sp2 - [18-19] with 4\sp{}s or 5-5+ shape, relay to 2NT.  Rebids:
                        \li \qquad \cl3 - big 5-5 in \he{}/\cl{}
                            \li \qquad \qquad \di3 - club fit, denies \he{} fit
                        \li \qquad \di3 - big 5-5 in \he{}/\di{}
                        \li \qquad \he3 - 4\sp{}s, \cl{} splinter
                        \li \qquad \sp3 - 4\sp{}s, \di{} splinter
                        \li \qquad 3NT - [18-19], 4522 (semi-balanced)
                        \li \qquad \cl4 - big 5-6 in \he{}/\cl{}
                        \li \qquad \di4 - big 5-6 in \he{}/\di{}
                    
                    2NT - [18-19] balanced.  Denies 4\sp{}s
                    \li \cl3 - [17-21] with exactly 4\cl{}s
                        \li \qquad \di3 - stopper ask
                        \li \qquad \he3 - possible fit, 2\he{}s.  May have 3\he{}s if [12+].
                        \li \qquad \sp3 - 6+\sp{}s
                        \li \qquad 3NT - to play, typically a stopper

                    \li \di3 - [17-21] with exactly 4\di{}s.  Use above continuation with \cl4 as cue
                    \li \he3 - 6+\he{}s, single suited.  3\sp{} promises 6+, 3NT denies a fit.
                    \li \sp3 - [20-21], 4\sp{}s
                }
                \li \he2 - [6-7] 2\he{}s
                \li \sp2 - [6-7] 1-\he{}s, 5+\sp{}s, or 6+ with all HCP in \sp{}
                \li 2NT - minors, typically longer diamonds (thus 4144 or 4054)
                \li \cl3 - [6-7] 5+\cl{}s (typically 6).
                \li \di3 - [6-7] 5+\di{}s (typically 6).
            }
            \li \di2 - [11-16] with 4+\di{}s, promises a singleton/void unless [14] balanced.
            \ul {
                \li \he2 - signoff
                \li \sp2 - signoff
                \li 2NT - invite
                \li \cl3 - 4th suit forcing
                \li \di3 - [9-11] 4+ \di{}s, invite to 3NT
                \li \he3 - [11-12] 3-card limit raise
                \li \sp3 - [9-11] 6+\sp{}s
            }
            \li \he2 - [11-14] with 6+\he{}s
            \li \sp2 - [11-16] with 4\sp{}s
            \li 2NT - [14(+)-16] with 4\sp{}s, splinter
            \ul {
                \li \cl3 - singleton ask, invitational+
                \ul {
                    \li \di3 - in \di{}s
                    \li \he3 - in \cl{}s (surrogacy principle)
                }
                \li 3\di{}/\he{} - cue
                \li \sp3 - to play
            }
            \li \cl3 - [14(+)-16] with 5-5 in \cl{}/\he{}
            \ul {
                \li \di3 - flag \cl{}s
                \li \he3 - to play
                \li \sp3 - flag \he{}s
                \li 3NT - to play
            }
            \li \di3 - [14(+)-16] with 5-5 in \di{}/\he{}
            \ul {
                \li \he3 - to play
                \li \sp3 - flag \di{}s
                \li 3NT - to play
                \li \cl4 - flag \he{}s
            }
            \li \he3 - [14(+)-16] with 6+\he{}s
            \ul {
                \li \sp3 - 6+\sp{}s, forcing
                \li 3NT - to play
                \li 4m - fit, cue
            }
            \li \sp3 - [14(+)-16], 4522.  Flag \sp{}s
            \ul {
                \li 3NT - non-serious, forcing
                \li 4m/\he{} - cue
                \li \sp4 - to play
            }
            \li \cl4 - [14(+)-16] 6-5 in \cl{}/\he{}.  Kickback on for both
            \li \di4 - [14(+)-16] 6-5 in \di{}/\he{}
        }

        \li 1NT - invitational-, denies 4+\he{}s.  Non-forcing
        \ul {
            \li \cl2 Gazzilli
            \ul {
                \li \di2 - [8+] any
                \ul {
                    \li \sp2 - relay to 2NT, 5-6 in majors or big 5-(5+) hand.  Rebids:
                        \li \qquad \sp3 - 5-6 in majors
                        \li \qquad \cl4 - 6-5 in \cl{}/\he{}. Kickback on for both
                        \li \qquad \di4 - 6-5 in \di{}/\he{}
                    \li 2NT - [18-19] balanced or semi-balanced
                }
                \li \sp2 - [6-7] (44) or better in minors, equal or longer clubs
            }
        }
        \li \cl2 - 4+\cl{}s, 2/1 GF (elaborated in \sp1 section)
        \li \di2 - 4+\di{}s, 2/1 GF
        \li \he2 - [8-10] 3+ \he{}s, constructive raise
        \ul {
            \li \sp2 - invitational ambiguous splinter.  Relay to 2NT or bid major with min/max.
            \ul {
                \li 2NT-\he3 - spade shortness (surrogate)
            }
            \li 2NT - Help suit in spades
            \li \cl3 - [15-16] Help suit in clubs
            \li \di3 - [15-16] Help suit in diamonds
            \li \he3 - [15-16] informationless invite
            \li \sp3 - [18+] splinter
            \li 3NT - pass-or-correct
        }

        \li \sp2 - [0-4] 6+ \sp{}s, very weak
        \li 2NT - 4+\he{}s GF, balanced unless [17+]
        \ul {
            \li \cl3 - shortness
            \li \di3 - shortness
            \li \he3 - [18+]
            \li \sp3 - shortness
            \li 3NT - [15-17] semi-balanced submaximum 
            \li 4m - side 5-card suit
            \li \he4 - [11-14] minimum
        }
        \li \cl3 - [10-12] with 4+\he{}s or [12-13] splinter
        \ul {
            \li \di3 - artificial asking (invitational or slammish)
            \ul {
                \li \he3 - [10], garbage bad hand
                \li \sp3 - [12-13] shortness
                \li \cl4 - [12-13] shortness
                \li \di4 - [12-13] shortness
                \li \he4 - [10(+)-12], accept
            }
            \li \he3 - to play
            \ul {
                \li \he4 - [12-13] ambiguous splinter
            }
        }
        \li \di3 - [7-9] with 4+\he{}s, constructive
        \li \he3 - [0-6] with 4+\he{}s, preemptive
        \li \sp3 - [14-16] splinter (weaker splinters go through \cl3)
        \li 3NT - [13-16] (4)3(33), pass-or-correct
        \li \cl4 - [14-16] splinter
        \li \di4 - [14-16] splinter
        \li \he4 - [4-9] 5+\he{}s, not balanced unless favorable
        \li \sp4 - [4-9], to play.  Flag \sp{}s\n\n
    }

    Passed hands use natural 1NT/\di2/2NT, \cl2 Drury, and support jump shifts:
    \ul {
        \li 1NT - [6-10] no fit
        \li \cl2 - [10-11] 3+\he{}s.  Drury, denies a support jump shift
        \ul {
            \li \di2 - [13], looking for a maximum.  Could be slammish.
            \li \he2 - Signoff
        }
        \li \di2 - [10-11] 5+\di{}s.  Denies a fit
        \li \he2 - [6-9] 3+\he{}s.
        \li \sp2 - [10-11] 5+\sp{}s, 4\he{}s.
        \li 2NT - [11-12] no fit, invite.
    }
\newpage

\bidsection{\sp{1}}{1♠} \label{2:5}

    Unlike \he1, \sp1 has simple rebids as responder either shows a fit with defined strength through a myriad of raises, or immediately limits their hand via semi-forcing 1NT.\n

    Gazzilli is still played, though a few adapatations are necessary.  As such, only the different sequences will be mentioned.  Use as much of the \he1 structure as possible.\n

    Since responder may have 4 hearts, the 2NT bid is used to show 6-4 in the majors exactly.  Passed hand structure is identical.
    
    The structure is as follows:

    \ul {
        \li 1NT - invitational-, denies 4\sp{}s.
        \ul {
            \li \cl2 - Gazzilli
            \ul {
                \li \di2 - [8+]
                \ul {
                    \li \he2 - relay to \sp2, big 5-5
                }
                \li \he2 - [6-7] 5+ \he{}s, 1-\sp{}
            }
            \li \di2 - [11-16] 4+\di{}s
            \li \he2 - [11-16] 4+\he{}s
            \li \sp2 - [11-14] 6+\sp{}s
            \li 2NT - [14(+)-16] 6-4 in the majors
            \li \cl3 - [14(+)-16] 5-5 in \cl{}/\sp{}s
            \li \di3 - [14(+)-16] 5-5 in \di{}/\sp{}s
            \li \he3 - [14(+)-16] 5-5 in \he{}/\sp{}s
            \li \sp3 - [14(+)-16] 6+\sp{}s, single suited
        }
        \li \cl2 - 3+\cl{}s, 2/1 GF (3 only when 3433)
        \ul {
            \li \di2 - [11-21] 4+\di{}s (lower suits are free to show)
            \li \he2 - [11-21] 4+\he{}s
            \li \sp2 - [11-15] Default response
            \li 2NT - [14] or [18-19] balanced (show slammish strength later)
            \li \cl3 - [14+], 4+\cl{}s (raises show 16+ playing strength).  Flagged unless major is \textit{immediately} supported
            \li \di3 - [16+], 5-5 in \di{}/\sp{}s with great suit quality
            \li \he3 - [16+], 5-5 in \he{}/\sp{} with great suit quality
            \li \sp3 - [18+], 6+\sp{}s single suited.  Flagged
        }
        \li \di2 - 4+\di{}s, 2/1 GF
        \ul {
            \li \cl3 - [16+] 4+\cl{}s, typically unbalanced (reverses show HCP)
        }
        \li \he2 - 5+\he{}s, 2/1 GF
        \ul {
            \li \he3 - [14+] 3+\he{}s (slow shows).  Flags hearts
            \li \he4 - [11-13] 3+\he{}s (fast arrival, typically balanced)
        }
        \li \sp2 - [8-10] constructive raise
        \ul {
            \li 2NT - invitational ambiguous splinter.  Relay to \cl3 or bid major with min/max.
            \ul {
                \li \cl3- club shortness (surrogate)
            }
        }
        \li \he3 - [14-16] 4+\sp{}s, splinter
        \li \he4 - 7+\he{}s, preemptive jump
    }

    For 2/1 auctions, jumps above reverses are splinters such as: \he1-\cl2-\sp3.  In case of a splinter in minor, you sacrifice Kickback, which is accessible by setting trumps via a natural raise (\sp1-\di2-\he4 is a splinter, since \di3 confirms a fit for Kickback)
\newpage

\bidsection{1NT [(14+)-17]}{1NT [(14+)-17]} \label{2:6}

    Any 15-17 with (4333), (4432), or (5332) distribution are opened 1NT.  The semi-balanced hands (5422), (6322) may be opened 1NT only if the longest suit is a minor. 14 HCP hands with a strong 5+ suit may be upgraded to 1NT.  When responder range-asks, a minimum is [14(+)-15], and maximum are [16-17].  A 15-count with good controls and shape may be upgraded \textit{only} when responder is slammish.
    \\

    The main feature of this NT structure is exploring responder's shape while efficiently using bidding space.  As bridge is about the majors, most conventions are geared toward identifying 4-4 and 5-3 major fits.\n

    However, when responder and opener have a good minor fit with shape/points, it is desireable to be in a safer minor game or slam slam as opposed to 3NT or 6NT.  Knowing the size of a fit is important, so responder's first bids show their suit length. For slammish hands, this is the following structure: \n
    
    With a balanced hand without a major, responder can use \sp2 or \cl3 puppet followed by a quantitative jump to 4NT, after which the 4NT sequence is used (see below). \n
    
    For 6-card slammish minor hands, responder can transfer, splinter, and keycard.  With a 4-card major, use puppet and bid the minor.  Minors after puppet promise 6 cards. \n

    With unbalanced 5-card minor hands without a major, responder has \di3, \he3, and \sp3.  Thus, the last class of hands are 5431 hands with a 5-card minor and a 4-card major.  These are shown through \cl2 stayman, with a conventional sequence to show responders exact shape. \n\n


    The full structure is as follows:
    \ul{
        \li \cl2 - stayman (not 5-5 in majors) or slammish with \textit{exactly} 5 cards in a minor
        \ul{
            \li \di2 - no majors; South African transfers ON
            \ul{
                \li 2M - 5-card invite
                \li 2NT - invite
                \li \cl3 - slammish with 5\cl{}s
                \ul{
                    \li Step 1 - minimum, fit
                    \li suits - max, cue
                    \li 3NT - no fits
                    \li \cl4 - max, cue
                    \li \di4 - Kickback
                }

                \li \di3 - slammish with 5\di{}s. Use above continuation

                \li 3M - 5 in other major, GF (smolen)
            }
            \li \he2 - 4+\he{}s
            \ul {
                \li \sp2 - 5\sp{}s 4\he{}s invite
                \li 2NT - 4\sp{}s invite
                \li \cl3 - 3-card raise, 4 in other major.  Slammish with (15) in minors
                \ul {
                    \li \di3 - specify shape
                        \li \qquad Step 1 - 5\cl{}s, slam invite.  Cuebid 4m as 'transfer' to Kickback
                        \li \qquad Step 2 - 5\di{}s, slam invite (may be 3NT, passable)
                        \li \qquad Step 3 - 5\cl{}s, slam forcing (may be 3NT, not passable)
                        \li \qquad Step 4 - 5\di{}s, slam forcing

                    \li 3M - fit, slam interest
                    \li 4M - fit, no interest
                }
                \li \di3 - singleton or void in major.  40(45) or 41(35) shape.  Use above continuation
                \li \he3 - invite
                \li \sp3 - slammish \he{} raise
                \li 3NT - 4\sp{}s, pass or correct.
            }

            \li \sp2 - 4+\sp{}s, denies 4\he{}s.  Use above continuation
        }
        \li \di2 - 5+\he{} transfer
        \ul {
            \li \he2 - accept
            \ul {
                \li \sp2 - 5-5 majors invite
                \li 2NT - 5\he{}s invite
                \li \cl3 - 4+\cl{}s GF
                \ul {
                    \li \di3 - 4+\cl{}s, minimum, denies heart fit
                    \li \he3 - 3+\he{}s, with extras
                    \li \sp3 - 4+\cl{}s, maximum, denies heart fit
                    \li 3NT - no fit
                }
                \li \di3 - 4+\di{}s GF.  Use above continuation
                \ul {
                    \li \cl4 - 4+\di{}s, maximum
                }
                
                \li \he3 - 6+\he{}s invite
                \li \sp3 - splinter
                \li 3NT - pass or correct
                \li \cl4 - splinter
                \li \di4 - splinter
                \li \he4 - 6332 slam invite
                \li \sp4 - EKC 0314
                \li 4NT - quantitative
            }
            \li \he3 - superaccept; 5\he{}s or [16-17] 4\he{}s, not 4333
            \ul {
                \li Cues, Kickback, EKC on
            }
        }

        \li \he2 - 5+\sp{} transfer. Use above continuation from \di2
        \ul {
            \li \sp2 - accept
            \ul {
                \li \he3 - 5-5 majors GF
            }
        }

        \li \sp2 - 6+\cl{} weak, 6+\cl{} GF, or range-ask
        \ul {
            \li 2NT - [15]
            \ul {
                \li \cl3 - signoff
                \li \di3 - splinter
                \li \he3 - splinter
                \li \sp3 - splinter
                \li 3NT - to play
                \li \cl4 - good slam invite
                \li \di4 - Kickback
            }
            \li \cl3 - [16-17]. Use above continuation
        }

        \li 2NT - 6+\di{} weak, 6+\di{} GF, or 5-5\cl{}/\di{} very weak
        \ul {
            \li \cl3 - 2-\di{}s
            \ul {
                \li \di3 - signoff
                \li \he3 - splinter
                \li \sp3 - splinter
                \li 3NT - to play, mild slam invite
                \li \cl4 - splinter
                \li \di4 - good slam invite
                \li \he4 - Kickback
            }
            \li \di3 - any other hand. Use above continuation
        }

        \li \cl3 - puppet stayman, or slammish 6+minor with 4-card major.

        \ul {
            \li \di3 - no 5-card major, does \textit{not} promise a major
            \ul {
                \li \he3 - 4\sp{}s
                \li \sp3 - 4\he{}s
            }
            \li \he3 - 5\he{}s
            \ul {
                \li \sp3 - forcing \he{} raise.
                \li 3NT - to play
                \li 4m - 6+ suit, natural slam invite.  Typically 4\sp{}s.
            }
            \li \sp3 - 5\sp{}s
        }
        \li \di3 - 5-5 or better in \cl{}/\di{} GF
        \ul {
            \li \he3 - flag clubs, cooperative
            \li \sp3 - flag diamonds, cooperative
            \li 3NT - double stops in both majors 44(32) or (53)(32)
            \li 4m - minimum, support for minor, typically with major wastage
        }
        \li \he3 - 31(45) GF.  Singleton not A/K unless slammish
        \ul {
            \li \sp3 - 4\sp{}s, looking for 4-3 fit
            \li 3NT - double stopper
            \li \cl4 - good clubs (or 33)
            \li \di4 - good diamonds
            \li \he4 - maximum, double minor fit
            \li \sp4 - 5\sp{}s, to play
        }
        \li \sp3 - 13(45) GF
        \li 3NT - to play
        \li \cl4 - 6+\he{}; transfer to play, 1430, or cuebid
        \li \di4 - 6+\sp{}; transfer to play, 1430, or cuebid
        \li \he4 - to play
        \li \sp4 - to play
        \li 4NT - Quantitative (forcing if opener already showed range)
        \ul {
            \li \cl5 - 4\cl{}s, denies 4\di{}s
            \li \di5 - 4\di{}s, denies 4\cl{}s
            \li 5NT - 4-4 or better in minors
            \li \cl6 - 5\cl{}s
            \li \di6 - 5\di{}s
            \li 6NT - to play
        }
    }
\newpage
\bidsection{\cl{2} - Game-Forcing}{2♣ - Game-Forcing} \label{2:7}

    \cl2 is the strongest opening bid, showing a game forcing hand or [22-24] balanced (which should almost always be raised to game anyway).  \cl2 has waiting and positive responses, Kokish \he2, cheaper minor, \di3 stayman, and conventional 3M jump rebids by opener.\n
    
    The structure is as follows:

    \ul {
        \li \di2 - no positive response, a positive NT, or a positive diamond with a 4-card major.
        \ul {
            \li \he2 - 5+\he{}s or [25+] balanced.  Forced relay to \sp2.
            \li \sp2 - 5+\sp{}s
            \li 2NT - [22-24] balanced
            \li \cl3 - 5+\cl{}s
            \ul {
                \li \di3 - stayman
            }
            \li \di3 - 6+\di{}s.  No 4-card major
            \li \he3 - 4\he{}s, 5+\di{}s
            \ul {
                \li \sp3 - forcing heart raise
                \li 3NT - to play
                \li \cl4 - cue for diamonds (denies hearts)
                \li \di4 - cue for diamonds (denies hearts)
                \li \he4 - to play
            }
            \li \sp3 - 4\sp{}s, 5+\di{}s
            \ul {
                \li \he4 - forcing spade raise (just like stayman, use other major)
            }
        }
        \li \he2 - 5+\he{}s, KQ or better
        \ul {
            \li \sp2 - 5+\sp{}s
            \li 2NT - [22-24] balanced.  Bid naturally
            \li \cl3 - 5+\cl{}s
            \li \di3 - 5+\di{}s
            \li \he3 - Flag suit.
        }

        \li \sp2 - 5+\sp{}s, KQ or better.  Use \he2 structure.
        \li \cl3 - 5+\cl{}s, KQ or better
        \ul {
            \li \di3 - stayman, or \di{} single suitor
            \li \he3 - 5+\he{}s
            \li \sp3 - 5+\sp{}s
            \li 3NT - [22] dead minimum, no fit
            \li \cl4 - 3+ \cl{}s, fit.  Cues preferable over keycard
            \li \di4 - Kickback
        }
        \li \di3 - 6+\di{}s, KQ or better.  5-card suit acceptable with 3-3 in majors.  Use \cl3 structure.
    }
\newpage

\bidsection{2\di{}/\he{}/\sp{}}{2♢/♡/♠} \label{2:8}

    The first of the weak-2 opening bids, \di2 is perhaps the most interesting.  It eats up bidding space without giving the opposition a major to look for.  \he2/\sp2 have the benefit of more often reaching \he4/\sp4, which makes it easier to reach game for the partnership.  Regardless of suit, it is important for the opener to have a well-defined hand for when their partner is preempted.  Since preempts depending on both vulnerability and seat, the definitions are as follows:

    \ul {
        \li non-vulnerable
        \ul {
            \li 1st seat - [4-10], JTxxxx or better
            \li 2nd seat - [6-10], requires a good feature of the hand
            \li 3rd seat - [4-10], (no restriction on quality/shape)
            \li 4th seat - [10-14], worse than \di1-\di2
        }
        \li vulnerable
        \ul {
            \li 1st/3rd seat - [5-10], QJT or better
            \li 2nd seat - [8-10], very nice preempt
            \li 4th seat - [10-14], worse than \di1-\di2
        }
    }

    When opening a preempt, one must consider their major holding.  Opposite an unpassed partner, a preempt should not contain another 4-card suit, especially a major (exceptions being both minors or very weak).  When opening a major, it is a liability to hold 3-card support for the other major to avoid missing 5-3 fits.  When responder has a fit and wishes to preempt further, they may elect to make a non-forcing raise (RONF).  All other actions are forcing and invitational+.\n
    
    After responder bids a new suit, opener should retreat to their suit with a [4-6] non-fit, raise once with a [4-6] and 3-card support, raise twice with [7-10] 3-card fit, and cue a feature otherwise.

    \ul {
        \li \sp2 - [17+] 5+\sp{}s
        \ul {
            \li \cl3 - [7-10] feature, no fit
            \li \di3 - [7-10] feature, no fit
            \li \he3 - [4-6] minimum, no fit
            \li \sp3 - [4-6] minimum, fit
            \li \sp4 - [7-10] maximum, fit
        }
        \li 2NT - OGUST
        \ul {
            \li \cl3 - [4-7] JTxxxx or worse (1-/3)
            \li \di3 - [4-7] Qxxxxx or better (2/3)
            \li \sp3 - [8-10] Qxxxxx or worse (1-/3)
            \li \sp3 - [8-10] KQxxxx or better (2/3)
            \li 3NT - AKQxxx (3/3)
        }
        \li \cl3 - [17+] 5+\cl{}s
        \ul {
            \li \he3 - [4-6] minimum
            \li \sp3 - [7-10] no fit.  Note opener cannot bypass 3NT
            \li \cl4 - [4-6] fit
            \li \cl5 - [7-10] fit
        }
        \li \di3 - [17+] 5+\di{}s
        \li \he3 - [0+] with support (preemptive)
        \li 3NT - to play
        \li \he4 - [0+] with support (preemptive or strong)
    }

    Note that point ranges are rough guidance. \hhand{KT9, x, QJT9xxx, xxx} is worth much more than [6] opposite a \sp2 response, while \hhand{xx, QJ, KJxxxx, QJx} is worth less than the [9] advertised.  New suits after OGUST are cuebids for the preempted suit, game-forcing and possible slam interest.
\newpage

\bidsection{2NT [20-21]}{2NT [20-21]} \label{2:9}

    2NT may be opened with shapes (4333), (4432), (5332), (5422), (6332), and (5431).  Singletons must be either A or K, opener must not have a 6-card major or 5-4 in the majors.\n
    Puppet and transfers are used to investigate major fits, and \sp3 is used for hands with both minors.  Since opener has more points than responder, 4M is \textbf{not} to play, and instead shows a six card minor (two-below) with slam interest (inspired from Scanian methods). The full structure is as follows:
    \ul {
        \li \cl3 - puppet stayman; minors show a 5+ (usually 6) card suit
        \ul {
            \li \di3 - at least one 4-card major
            \ul {
                \li \cl4 - natural, clubs
                \li \di4 - natural, diamonds
                \li \he4 - pass-or-correct ***

            }
        }
        \li \di3 - 5+\he{} transfer
        \ul {
            \li \he3 - accept
            \ul {
                \li \sp3 - ***
                \li 3NT - pass or correct
                \li \cl4 - 4+\cl{}s
                \ul {
                    \li \di4 - 4+\cl{}s, denies \he{} support
                    \li \he4 - 3+\he{}s
                    \li \sp4 - ***
                    \li 4NT - no support
                }
                \li \di4 - 4+\di{}s
                \li \he4 - mild heart slam try, typically (6331)
                \li \sp4/5m - EKC
                \li 4NT - quantitative

            }
            \li \he4 - 4+\he{}s, superaccept (good outside tricks)
        }
        \li \he3 - 5+\sp{} transfer.  Use above continuation
        \li \sp3 - (54)+ in minors
        \li 3NT - to play
        \li \cl4 - transfer to \he4, to play, cuebid, or 1430
        \li \di4 - transfer to \sp4, to play, cuebid, or 1430
        \li \he4 - 6+\cl{}s (usually 7), slammish.  Cuebids on
        \li \sp4 - 6+\di{}s (usually 7), slammish. Cuebids on
        \li 4NT - [12], quantative
        \li 5m - to play
    }
\bidsection{3\cl{}/\di{}/\he{}/\sp{}}{3♣/♢/♡/♠} \label{2:10}

    Unlike 2-level preempts, 3-level preempts are mostly non-constructive.  They almost always show a 7-card suit; the exceptions occur 3rd seat at favorable vulnerability or \cl3 (since \cl2 is not preemptive).  Since 4m is artificial, 3-level minor preempts may have an 8-card suit, though 8-card suits may also be opened at the 5-level. \n

    If opener preempted \cl3 or \di3, majors are natural and forcing (typically 6-cards).  Barring 3m-3M, suits are cues/Kickback/splinters, slam invitational or better.

\bidsection{3NT - Gambling}{3NT - Gambling} \label{2:11}

    3NT shows a 7-card (or better) minor headed by AKQ (or better).  In 1st/2nd seat, opener may have no more than an outside Q.  In 3rd/4th seat, opener promises \textit{exactly} one outside stopper.\n

    Responder may correct to any level of minor (pass or correct) or cue a major.

\bidsection{4\cl{}/\di{}/\he{}/\sp{} - NAMYATS}{4♣/♢/♡/♠ - NAMYATS} \label{2:12}

    The 4-level openings are reserved for 8-card major preempts.  Direct 4M is weak, showing [4-10] HCP and an 8-card suit, typically broken unless weak on the outside suits.\n

    \cl4 and \di4 are transfers to \he4 and \sp4 respectively.  These show a self-sufficient suit (3/4 honors; 7+ tricks) and outside values (8 tricks total).\n

    Accepting the transfer is signoff.  Bidding the intermediate suit is an artificial asking bid.  Retreat to 4M shows a minimum 8 tricks, any further bid is a cuebid.\n

    Kickback is on by responder; cues on for both.  The schema is as follows:

    \ul {
        \li \cl4 - NAMYATS, 8-card or better heart suit and 8+ tricks
        \ul {
            \li \di4 - Artificial asking bid
            \ul {
                \li \he4 - minimum 8 tricks
                \li \sp4 - spade cue, 8.5+ tricks
            }
            \li \sp4 - Kickback
            \li 4NT - spade cue
            \li \cl5 - club cue
        }
        \li \he4 - Weak preempt, 8-card broken heart suit or less than 8 tricks.  Use above continuation
    }

\chapter{Interference}
\section{Overview} \label{3:1}

Unfortunately, bridge is not a simple as a back-and-forth conversation with our partner.  Thus, we must have methods for competitive auctions, both when our opponents preempt and overcall.  Since both opener's and overcaller's hands have a suit, our methods show the probable and important hands of responder.  Bids in parenthesis are interference: \cl1-(\he1)-X-(XX). \n

The following conventions are used:
\begin{itemize} \itemsep0em
    \item Takeout, negative, support, optional, reopening X
    \item Penalty, support, value XX
    \item Transfers over 1M-(X)
    \item Jordan 2NT
    \item lebensohl
    \item `Lower-lower' cues
\end{itemize}
Contents:
\ul {
    \li \ref{3:1} - Overview
    \li \ref{3:2} - Suit-based Interference - 2nd Seat
    \li \ref{3:3} - Suit-based Interference - 4th Seat
    \li \ref{3:4} - NT Interference
    \li \ref{3:5} - Preempts
    \li \ref{3:6} - Two-Suited Overcalls
}

\newpage
\section{Suit-based Interference - 2nd Seat} \label{3:2}
    
    Over an opposing double, responder may pass with [0-6] points and no fit, as a strong opener may clarify a strong hand later.  With [7+], responder owes a response.  XX shows balanced [10+], without a 9-card fit.  If a minor was opened, new suits are natural freebids, and 1NT is natural; if a major was opened, 1NT-2(M-1) are transfers, with 2M as a preemptive raise.\n

    Additionally, jump shifts show a quality 5-card suit with 4-card support and constructive values.  2NT shows a 9-card fit and invitational+ values (Jordan 2NT).  Suits above a jump shift are game-forcing splinters.  3NT is natural, typically showing majors stopped. \n

    Over a suit overcall, X is negative, and freebids are natural invitational+, always forcing.  Cues show invitational+ with support.  Jumps are preemptive.\n

    Here is the common auctions against direct suit-based overcalls:
    
    \ul {
        \li \cl1-(X)
        \ul {
            \li pass - [0-6]
            \li XX - [10+] no 9-card fit.  Typically a natural 2NT response.  Penalty X on
            \li 1z - [7+] 4-card suit or better (system on)
            \li 1NT - [7-9] balanced, no majors
            \li \cl2 - [7-9] 4+\cl{}s
            \li \di2 - [7-10] quality 5+\di{}s, 4\cl{}s
            \li \he2 - [7-10] quality 5+\he{}s, 4\cl{}s
            \li \sp2 - [7-10] quality 5+\sp{}s, 4\cl{}s
            \li 2NT - [11+] 5+\cl{}s
            \li \cl3 - [4-7] 4+\cl{}s, preemptive
            \li 3z - 5+\cl{}s, splinter.
        }

        \i \cl1-(\di1)
        \ul {
            \li pass - [0-6] or [10+] with interest in penalizing \di1
            \li X - \textit{exactly} 4-4 in \he{}/\sp{}
            \li \he1 - [7+] 4+\he{}s, system on
            \li \sp1 - [7+] 4+\sp{}s, system on
            \li 1NT - [7-10] typically a stopper
            \li \cl2 - [7-10] 4+\cl{}s
            \li \di2 - [11+] 4+\cl{}s
            \li 2M - [4-7] 6-card suit, preemptive
            \li 2N - [11-12] with a stopper
            \li \cl3 - [4-6] 5+\cl{}s, weak
            \li 3M - [4-7] 7-card suit, preemptive
        }

        \li \cl1-(\he1)
        \ul {
            \li X - [7+] \textit{exactly} 4\sp{}s, or GF balanced without a stopper (cue later)
            \li \sp1 - [7+] 5+\sp{}s
            \li \di2 - [11+] 5+\di{}s
        }

        \li \cl1-(\sp1)
        \ul {
            \li X - [7+] 4\he{}s, or [7-10], 5+\he{}s (too weak for a freebid)
        }

        \li \he1-(X)
        \ul {
            \li XX - [10+] may have 3-card support
            \li \sp1 - [7+] 4+\sp{}s
            \li 1NT - [4-7] 6+\cl{}s or [8-9] 5+\cl{}s
            \ul {
                \li \cl2 - [11-17] 1+\cl{}
                \ul {
                    \li pass - [4-7] 6+\cl{}s
                    \li \he2 - [8-9] 2\he{}s, 5+\cl{}s
                    \li \cl3 - [8-9] 6+\cl{}s, quality suit
                }
                \li \di2 - [11-17] 4+\di{}s, void in clubs
                \li \he2 - [11-17] 6+\he{}s
                \li \sp2 - [18+] artificial.  Natural reverse, 6+\he{}s GF, or 3+\cl{}s GF.
                \li 2NT - [18-19] balanced
                \ul {
                    \li \cl3 - [4-5] 6+\cl{}s, signoff
                }
                \li \cl3 - [14-17] 3+\cl{}s
                \li \di3 - [19+] 4+\di{}s, game-forcing
            }
            \li \cl2 - [4-7] 6+\di{}s or [8-9] 5+\di{}s
            \li \di2 - [7-10] 3\he{}s, constructive
            \li \he2 - [0-6] 3\he{}s, preemptive raise
        }
    }

    Often, responder is too weak, balanced, or strong in the overcalled suit to respond.  Thus, opener should try to reopen with X with almost any hand, as it would be distastrous for our opponents to play \sp1-4 for -400 instead of -1100 when responder holds \spades{AQT872} over overcaller has \spades{KJ965}. \n
    
    With strong hands, simply make the reverse bid or NT cheaply to show strength.  Unfortunately, advancer may raise.  Again, all actions by opener are strong, but a weak responder needs an escape via lebensohl.  Some examples:

    \ul {
        \li \cl1-(\sp1)-p-(p)
        \ul {
            \li X - [12+] short in spades (almost all hands)
            \li 1NT - [18-19] balanced with stopper
            \li \di2 - [18+] natural reverse
            \li \he2 - [18+] natural reverse
        }
        \li \di1-(\he1)-p-(\he2)
        \ul {
            \li X - [17+] takeout
            \ul {
                \li \sp2 - signoff
                \li 2NT - lebensohl, puppet to \cl3 for signoff
                \li 3z - [6-8], invitational
            }
        }
    }
\newpage
\section{Suit-based Interference - 4th Seat} \label{3:3}

    When the 4th suit overcalls a suit after in auction x-y-(z), opener already has shown [11+] strength and responder is able to bid.  Since all bids (including pass!) show [11+], bids can be more descriptive.\n

    Importantly, we have access to support X (or XX), showing \textit{exactly} 3-card support for responder's suit.  This is on through (z)=(\he2), as the partnership must be able to stop at the 2-level in a 4-3 fit.  For \sp2 and higher, double by opener are strong, typically denying support and penalty-oriented.  Some sample sequences (identical bids are ignored, \cl1-\he1-(\sp1)-\di3 is still a mini-splinter):

    \ul {
        \li \cl1-\he1-(\sp1)-
        \ul {
            \li X - [11-21] \textit{exactly} 3-card support.  \textbf{All other bids deny 3-card support}
            \li pass - no other bid, typically [12-14] balanced, no stopper
            \li 1NT - [12-14] balanced with spade stopper
            \li \sp2 - [18-19] balanced without a stopper
            \li 2NT - [18-19] balanced with a stopper
        }

        \li \di1-\sp1-(\he2)-
        \ul {
            \li X - support double
            \li pass - no other bid
            \li \sp2 - [11-14] 4\sp{}s
            \li 2NT - [18-19] balanced with a stopper
            \li \he3 - [18-19] without a stopper (pushy but necessary)
        }

        \li \he1-1NT-(\sp2)
        \ul {
            \li X - [18+] takeout
            \li 2NT - [18-19] with stopper
            \li \cl3 - [18+]
        }
    }

\newpage
\section{NT Interference} \label{3:4}

    Occasionally after 1-suit, our opponents will overcall at natural 1NT.  Our responses are as follows:

    \ul {
        \li \di1-(1NT)
        \ul {
            \li X - [9+] penalty.  All further X for penalty
            \li \cl2 - [6-9] 5+\cl{}, quality suit
            \li \di2 - [6-9] 4+\di (a fit)
            \li \he2 - [6-9] 6+\he{}s, quality suit
            \li \sp2 - [6-9] 6+\sp{}s, quality suit
        }
    }

    With distribution and values, good opponents will compete over our 1NT openings.  Over (X) and (\cl2), system is on, unless they show both majors.  Our XX is penalty oriented, X by responder shows values takeout-oriented though opener may X for penalties after.  If opponents show both majors, our defense is as follows:

    \ul {
        \li X - values, penalty interest
        \li \di2 - natural, signoff
        \li \he2 - [5-8] (45) or better in minors
        \li \sp2 - [8-9] (45) or better in minors
        \li 2NT - lebensohl, forced puppet to \cl3.  Rebids:
        \ul {
            \li \di3 - signoff
            \li \he3 - stopper
            \li \sp3 - stopper
        }
        \li \cl3 - 5+\cl{}s, game forcing
        \li \di3 - 5+\di{}s, game forcing
    }

    When our opponents show a single suitor or two-suitor, 2-level rebids are natural and signoff, 2NT is lebensohl, and 3-level bids show a 5-card suit and game-forcing values.  An example after 1NT-(\he2) showing hearts and a minor:

    \ul {
        \li X - [8-9] 4\sp{}s
        \li \sp2 - 5+\sp{}s, signoff
        \li 2NT - lebensohl.  5\sp{}s invite, weak minor, or 4\sp{}s.  After \cl3 puppet:
        \ul {
            \li \di3 - signoff
            \li \he3 - 4\sp{}s, GF with a stopper (slow shows)
            \li \sp3 - 5\sp{}s, invitational
        }

        \li \cl3 - 5+\cl{}s, GF
        \li \di3 - 5+\di{}s, GF
        \li \he3 - 4\sp{}s, GF without a stopper
        \li \sp3 - 5\sp{}s, GF
        \li 3NT - to play
    }

    If 1NT is doubled for penalty (either as a 1NT overcall or an opening) the following escape is used:

    \ul {
        \li pass - forces opener to bid a 5-card suit or XX.  After XX: 
        \ul {
            \li pass - to play 1NTxx
            \li \cl2 - 4\cl{}s and higher suit
            \li \di2 - 4\di{}s and higher suit
            \li \he2 - 4\he{}s and a higher suit
        }
        \li XX - 5+ minor, transfer to \cl2
        \li \di2 - 5+\he{}s
        \li \he2 - 5+\sp{}s
    }
\newpage
\section{Preempts} \label{3:5}

    Preempts are the some of most obstructive bids opponents can make; they remove valuable bidding space when we have a invitational or strong auction.  As such, we must sacrifice some slam accuracy in favor of reaching as many making games as possible.\n

    After a preempt, negative doubles are on until 4\he{}s, though 4-level negative doubles are passable (optional doubles).  Freebids at the 2-level are invitational and game-forcing at the 3-level, promising a 5-card suit. NT responses show a stopper, denying the majors.  Cues show support and game-forcing values. \n

    A cue of a 2-level preempt primarily asks for a stopper (Western), while a cue of a 3-level preempt shows support and slam aspirations.  Jump reponses are game-forcing and show a strong 6-card suit, denying support for the opener's suit.  Some examples:

    \ul {
        \li \di1-(\sp2)
        \ul {
            \li X - [9+] 4+\he{}s
            \li 2NT - [11-12] with a stopper
            \li \cl3 - [12+] 5+\cl{}s, game-forcing
            \li \di3 - [7-10] 4+\di{}s
            \li \he3 - [12+] 5+\he{}s, game-forcing
            \li \sp3 - [12+] stopper ask or club support
            \li 3NT - to play, stopper
        }
        \li \he1-(\sp3)
        \ul {
            \li X - [12+] minors, more of a stopper ask
            \li 3NT - to play, stopper
            \li \cl4 - 5+\cl{}s, typically 6.
            \li \di4 - 5+\di{}s, typically 6.
            \li \he4 - to play
            \li \sp4 - Kickback (no cue available to set suit)
        }

        \li \cl1-(\he4)
        \ul {
            \li X - penalty
            \li \sp4 - 5+\sp{}s, to play
            \li 4NT - slam invite in \cl{}
            \li \cl5 - to play
            \li \di5 - to play
        }
        
        \li \sp1-(\di3)
        \ul {
            \li X - [9+] 4\he{}s
            \li \he3 - [12+] 5+\he{}s
            \li \sp3 - [8-10] 3+\sp{}s
            \li 3NT - to play, stopper
            \li \cl4 - 5+\cl{}s
            \li \di4 - slammish in \sp{}s
            \li \he4 - ***
            \li \sp4 - [11-15] 3+\sp{}s
        }
    }

    Sometimes, responder may not have the values or shape, and must pass.  At other times, they may have a 5 cards (or excellent 4 cards) in the preempted suit, interested in penalizing a preempt (and unable to bid 3NT).  Thus, opener should stretch to reopen with a X, around a good [13+].  As with the auction 1x-(1y)-p-(2y)-X, lebensohl is used.  An example:

    \ul {
        \li \di1-(\he2)-p-(p)-X
        \ul {
            \li pass - penalty
            \li \sp2 - [0-7] signoff
            \li 2NT - [0-7], puppet to \cl3 for signoff
            \li \cl3 - [8-11] 5+\cl{}
            \li \di3 - [4-7] 4+\di{}
        }
    }
\newpage
\section{Two-Suited Overcalls} \label{3:6}

    Whether in 2nd or 4th seat, our opponents can make 2-suited overcalls.  Since `our' suits and `their' suits are known, competition can become fierce with both sides potentially having a double fit so our methods must adapt. \n
    
    After a second seat two-suited overcall (Michaels, Unusual 2NT, another convention), pass shows [0-7], X shows [9+] with penalty interest, direct raises are competitive/preemptive, while cuebidding follows the `lower-lower' principle (a lower cuebid shows a lower suit). In order to bid 3NT, responder must have both suits stopped, so with only one, X and then cuebid the controlled suit. Some examples:

    \ul {
        \li \cl1-(\cl2) as both majors
        \ul {
            \li X - [9+] penalty suggestion, or forcing with 5+\di{}s
            \li \di2 - [6-9] 5+\di{}s (usually 6)
            \li \he2 - [10+] 4+\cl{}s (\he{} $<$ \sp{} and \cl{} $<$ \di{})
            \li \sp2 - [10+] 5+\di{}s, invitational
            \li \cl3 - [7-10] 4+\cl{}s, competitive
            \li \di3 - [0-9] 6+\di{}s, preemptive
            \li 3NT - to play, 1.5+ stoppers.
            \li \cl{4/5} - [0-9] 5+\cl{}s, preemptive raise
        }
        \li \di1-(2NT) as \cl{}/\he{}
        \ul {
            \li X-(\cl3)-p-p
            \ul {
                \li X - penalty
                \li \he3 - \he{} stopper, no \cl{} stopper
            }
            \li X-(\he3)-p-p
            \ul {
                \li \sp3 - \cl{} stopper, no \he{} stopper
            }
            \li \cl3 - [10+] 4+\di{}s (\di{} $<$ \sp{})
            \li \di3 - [7-10] 4+\di{}, competitive
            \li \he3 - [10+] 5+\sp{}s, invitational
            \li \sp3 - [7-10] 6+\sp{}s, competitive
        }
        \li \sp1-(\sp2) as \he{}/minor
        \ul {
            \li X - penalty or [10+] no fit
            \li 2NT - [11-12] \he{} stopper
            \li 3m - [7-9] 5+ cards (typically 6)
            \li \he3 - [10+] 3+\sp{}s
            \li \sp3 - [7-9] 3+\sp{}s, competitive
            \li 3NT - [12-15] \he{} stopper
            \li 4z - [12+] 4\sp{}s, splinter
            \li \sp4 - [0-9] 4\sp{}s, preemptive
        }
    }

    When a suited overcall occurs in fourth seat, methods are similar.  Support X/XX are on (overcalls of \sp2 or higher are penalty-oriented), direct cues show a stopper and [17-19], 2NT shows [17-19] with both suits stopped (typically balanced).  Some examples:

    \ul {
        \li \cl1-\he1-(X)
        \ul {
            \li pass - no other bid, typically [12-14] balanced
            \li XX - \textit{exactly} 3\he{}s
            \li 1NT - [12-14], partial stoppers in both suits
            \li \di2 - [17-19] diamond stopper
            \li \sp2 - [17-19] spade stopper
            \li 2NT - [17-19] both stopped
            \li other - same as \cl1-\he1-?
        }
        \li \cl1-\sp1-(2NT)
        \ul {
            \li pass - [12-16] no other bid
            \li X - [14+] penalty-suggestion
            \li \cl3 - [15-17] 6+\cl{}s
            \li \di3 - [17+] stopper
            \li \he3 - [17+] stopper
            \li \sp3 - 4\sp{}s, [15-17]
            \li 3NT - [17+] to play, both suits stopped
            \li \cl4 - [18+] 7+\cl{}s
            \li \di4 - [18+] 4\sp{}s, splinter
            \li \di4 - [18+] 4\he{}s, splinter
            \li \sp4 - [18-19] 4\sp{}s, balanced
        }
        \li \he1-1NT-(\he2)
        \ul {
            \li pass - no other bid, typically [12-14]
            \li X - [12-14] 6+\he{}s
            \li \sp2 - [18+] game force, typically no stopper
            \li 2NT - [18+] spade stopper
            \li \cl3 - [17+] 4+\cl{}s
            \li \di3 - [17+] 4+\di{}s
            \li \he3 - [15-17] 6+\he{}s
        }
    }
\chapter{Overcalling and Advancing}
\section{Overview} \label{4:1}
    
    When the opponents open the bidding, it is important to both interfere with their strong auctions, as well as reach games (and rarely, slams) when we have the points or shape.\n

    Since most players play some variation of 2/1, these methods are dedicated against 2/1.  At the end of this chapter, a section is dedicated to artificial or short \cl1 and \di1 openings. \n

    The following conventions are used:
    \begin{itemize} \itemsep0em
        \item Takeout, power, penalty, responsive, optional X
        \item Michael's, unusual 2NT, leaping Michael's
        \item Cappelletti, Multi-Landy
        \item Gambling 3NT / cues
        \item lebensohl\n
    \end{itemize}

    Expected strength:
    \begin{itemize} \itemsep0em
        \li (1y)-1z - [(7+)-17]
        \li (1y)-1NT - [15-18]
        \li (2y)-2NT - [16-19]
        \li (1y)-p-(p)-1NT - [11-15] \n
    \end{itemize}

    Contents:
    \ul {
        \li \ref{4:1} - Overview
        \li \ref{4:2} - Suit-based Overcalls - 2nd Seat
        \li \ref{4:3} - Suit-based Overcalls - 4th Seat
        \li \ref{4:4} - NT Overcalls and Defense
        \li \ref{4:5} - Preempts
        \li \ref{4:6} - Two-Suited Overcalls
        \li \ref{4:7} - Balancing
        \li \ref{4:8} - Unnatural \cl1 and \di1
    }

\section{Suit-based Overcalls - 2nd Seat} \label{4:2}
\section{Suit-based Overcalls - 4th Seat} \label{4:3}
\section{NT Overcalls and Defense} \label{4:4}
\section{Preempts} \label{4:5}
\section{Two-Suited Overcalls} \label{4:6}
\section{Balancing} \label{4:7}
\bidsection{Unnatural \cl1 and \di1}{Unnatural 1♣ and 1♢} \label{4:8}



\chapter{Carding}

\end{document}